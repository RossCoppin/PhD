% Options for packages loaded elsewhere
\PassOptionsToPackage{unicode}{hyperref}
\PassOptionsToPackage{hyphens}{url}
%
\documentclass[
]{article}
\usepackage{lmodern}
\usepackage{amssymb,amsmath}
\usepackage{ifxetex,ifluatex}
\ifnum 0\ifxetex 1\fi\ifluatex 1\fi=0 % if pdftex
  \usepackage[T1]{fontenc}
  \usepackage[utf8]{inputenc}
  \usepackage{textcomp} % provide euro and other symbols
\else % if luatex or xetex
  \usepackage{unicode-math}
  \defaultfontfeatures{Scale=MatchLowercase}
  \defaultfontfeatures[\rmfamily]{Ligatures=TeX,Scale=1}
\fi
% Use upquote if available, for straight quotes in verbatim environments
\IfFileExists{upquote.sty}{\usepackage{upquote}}{}
\IfFileExists{microtype.sty}{% use microtype if available
  \usepackage[]{microtype}
  \UseMicrotypeSet[protrusion]{basicmath} % disable protrusion for tt fonts
}{}
\makeatletter
\@ifundefined{KOMAClassName}{% if non-KOMA class
  \IfFileExists{parskip.sty}{%
    \usepackage{parskip}
  }{% else
    \setlength{\parindent}{0pt}
    \setlength{\parskip}{6pt plus 2pt minus 1pt}}
}{% if KOMA class
  \KOMAoptions{parskip=half}}
\makeatother
\usepackage{xcolor}
\IfFileExists{xurl.sty}{\usepackage{xurl}}{} % add URL line breaks if available
\IfFileExists{bookmark.sty}{\usepackage{bookmark}}{\usepackage{hyperref}}
\hypersetup{
  pdftitle={Quantifying passive kelp rafting phenomena and connectivity of nearshore communities model around the coast of South Africa:Development of a 3D spectral numerical coastal-flow},
  pdfauthor={RM Coppin; C Rautenbach; AJ Smit},
  hidelinks,
  pdfcreator={LaTeX via pandoc}}
\urlstyle{same} % disable monospaced font for URLs
\usepackage[margin=1in]{geometry}
\usepackage{longtable,booktabs}
% Correct order of tables after \paragraph or \subparagraph
\usepackage{etoolbox}
\makeatletter
\patchcmd\longtable{\par}{\if@noskipsec\mbox{}\fi\par}{}{}
\makeatother
% Allow footnotes in longtable head/foot
\IfFileExists{footnotehyper.sty}{\usepackage{footnotehyper}}{\usepackage{footnote}}
\makesavenoteenv{longtable}
\usepackage{graphicx,grffile}
\makeatletter
\def\maxwidth{\ifdim\Gin@nat@width>\linewidth\linewidth\else\Gin@nat@width\fi}
\def\maxheight{\ifdim\Gin@nat@height>\textheight\textheight\else\Gin@nat@height\fi}
\makeatother
% Scale images if necessary, so that they will not overflow the page
% margins by default, and it is still possible to overwrite the defaults
% using explicit options in \includegraphics[width, height, ...]{}
\setkeys{Gin}{width=\maxwidth,height=\maxheight,keepaspectratio}
% Set default figure placement to htbp
\makeatletter
\def\fps@figure{htbp}
\makeatother
\setlength{\emergencystretch}{3em} % prevent overfull lines
\providecommand{\tightlist}{%
  \setlength{\itemsep}{0pt}\setlength{\parskip}{0pt}}
\setcounter{secnumdepth}{-\maxdimen} % remove section numbering
% https://github.com/rstudio/rmarkdown/issues/337
\let\rmarkdownfootnote\footnote%
\def\footnote{\protect\rmarkdownfootnote}

% https://github.com/rstudio/rmarkdown/pull/252
\usepackage{titling}
\setlength{\droptitle}{-2em}

\pretitle{\vspace{\droptitle}\centering\huge}
\posttitle{\par}

\preauthor{\centering\large\emph}
\postauthor{\par}

\predate{\centering\large\emph}
\postdate{\par}

\title{Quantifying passive kelp rafting phenomena and connectivity of nearshore
communities model around the coast of South Africa:Development of a 3D
spectral numerical coastal-flow}
\author{RM Coppin\footnote{University of the Western Cape} \and C Rautenbach\footnote{South African Weather Service} \and AJ Smit\footnote{University of the Western Cape}}
\date{}

\begin{document}
\maketitle

{
\setcounter{tocdepth}{2}
\tableofcontents
}
\hypertarget{summary}{%
\section{Summary}\label{summary}}

Marine ecosystems are maintained by a variety of complex interactions
between abiotic and biotic variables such as temperature, wave exposure,
pH, competition, and processes such as top-down and bottom-up control,
predator-prey relationships and phenology (Doney et al. 2011; Harley et
al. 2012; Burrows et al. 2011; McGowan, Cayan, and Dorman 1998). These
abiotic and biotic variables, the interactions between them, and the
various ecological processes, ultimately determine the community
composition and ecological functioning of all ecosystems
({\textbf{???}}; Harley et al. 2006; Poloczanska et al. 2013; Jennings
and Brander 2010; Polovina 2005; Johnson et al. 2011; Krumhansl et al.
2016; Dayton et al. 1999). Climate directly and indirectly affects the
way in which abiotic and biotic variables interact, but is often
compounded by other impacts such as habitat destruction, pollution, and
over-fishing (Blamey and Branch 2012; Blamey et al. 2015). Temperature
and wave exposure have been recognised as important variables with
regards to climate-driven changes within the ocean (Guimaraes \&
Coutinho 1996; McGowan et al.~1998; McQuaid \& Branch 1984; Laufkötter
et al.~2015; Belkin 2009; Filbee-Dexter et al.~2018; Miller et al.~2011;
Smale et al.~2011; Smale \& Moore 2017). In order to persist and survive
within variable and changing environments, organisms must either
migrate, adapt, or die.

Seaweeds are sessile organisms that are unable to migrate to new areas
when local environmental conditions become unsuitable and therefore are
forced to adapt to new conditions in order to avoid expiration (Dayton
1985; Tegner et al.~1997). The main form of mortality for seaweeds is
through mechanical dislodgement by wave action (Schiel et al.~2006;
Seymour et al.~1989; Thomsen et al.~2004; Cavanaugh et al.~2011; Graham
et al.~1997; Demes et al.~2013; Edwards \& Estes 2006). Seaweeds,
particularly brown seaweeds, are able to undergo rapid morphological
adaptation to the hydrodynamic environment (Miller et al.~2011;
Friedland \& Denny 1995; Wing et al.~2007; Wernberg \& Thomsen 2005;
Fowler-Walker et al.~2006; Dudgeon \& Johnson 1992; Blanchette et
al.~2002). This allows seaweeds to reduce mortality through mechanical
dislodgement by inducing morphology which reduces overall drag. Seaweeds
that are unable to avoid mechanical dislodgement either raft out to sea
or wash up onto beaches. Not all beach-cast kelp may have originated
from a nearby kelp population and may have originated from other sites
or regions of the coast through rafting and ocean currents (Malm et
al.~2004; Emery \& Tschudy 1941; Filbee-Dexter \& Scheibling 2012;
Hobday 2000). Therefore, because kelp morphology is specific to its
local environment the morphological features may be able to indicate,
within a certain amount of probability, what site or region it most
likely originated from. In other words, beach-cast kelp may be used as
proxy for investigating the flow of coastal currents. Using kelp as a
proxy for determining its original location will be calibrated by means
of a hydrodynamic model which will be designed from already existing
SWAN and Delft3D models. This combined approach will allow investigation
into flow regimes around the west and south-west coasts of South Africa
and the role they play in subtidal ecology. Furthermore, the model could
be used to determine transport of plastic pollution around the coast.
For instance, micro-plastics are recognised as a threat to marine life
(Seltenrich 2015; Azzarello \& Van Vleet 1987; Wright et al.~2013;
Derraik 2002; Andrady 2011; Fendall \& Sewell 2009; Ivar Do Sul \& Costa
2014; O 'donoghue \& Marshall n.d.; Ballent et al.~2012), however very
little is known about how microplastics may be transported along the
coast.

The increased use of plastic in society over the past half century has
resulted in large amounts of plastic litter in both the marine and
terrestrial environment (Andrady 2011; Wright et al.~2013; Cole et
al.~2011). The problems associated with large plastic debris have
received attention for many decades, whereas those connected to marine
micro-plastics comparatively received very little attention. However,
today it has become a prioritized area among political organizations,
agencies and NGOs around the world. The micro-plastic debris present in
the ocean are derived from marine and terrestrial sources (Derraik 2002;
Wright et al.~2013; Cole et al.~2011; Seltenrich 2015), however there is
little understanding of how microplastics may be distributed with ocean
currents. Therefore, the coupling the adaptability of kelp morphology
and the ability to simulate hydrodynamic processes can greatly improve
our understanding of transport pathways and likely locations of
accumulation. This in turn may inform management decisions with regards
to eliminating and managing marine pollution in South Africa.

\hypertarget{background}{%
\section{Background}\label{background}}

Seaweeds, browns in particular, are capable of adapting morphological
characteristics to persist in changing and variable ocean environments.
Changes in morphology have been shown to alter photosynthetic ability,
nutrient uptake, and reduce probability of dislodgement with regards to
variation and changes in sea temperature , wave exposure and light
availability. Wave exposure has been shown to be an important driver of
seaweed morphology, as the main mechanism of seaweed mortality is
through the dislodgement. Changing morphology reduces drag and increases
the probability of survival. However, locally adapted seaweed may still
be dislodged in pulse disturbance events such as storms, and may raft
far distances and wash up on beaches. Beach-cast may not always
originate from adjacent kelp populations but rather from other regions
which the individual is adapted for. Therefore, the kelp morphology may
act as a proxy for investigating coastal currents and changes thereof.
The advances in ocean hydrodynamic modelling has made great progress and
has been applied in a variety of ways. This study will use advances in
hydrodynamic modelling in combination with kelp morphological
characteristics to investigate coastal currents. Once the model has been
established it may be applied in other ways, such as investigating the
transport of microplastics along the South African coastline. The harms
of microplastics to the marine environment has gained much traction in
recent years, but research in South Africa is lacking. The coupling of
kelp morphology and the ability to simulate hydrodynamic processes can
greatly improve our understanding of transport pathways and likely
locations of accumulation. This in turn may inform management decisions
with regards to eliminating and managing marine pollution in South
Africa.

Pollution is a huge environmental problem that affects both terrestrial
and marine ecosystems. Pollution from land enters the sea where it could
harm or kill marine organisms or is transported by ocean currents to
other coastal areas. In these areas the pollution could re-enter the
ocean or be blown by wind into terrestrial ecosystem where it could once
again be ingested by organisms causing harm or even death. Pollution is
therefore a major threat to the environment and marine organisms. In
recent years the effect of microplastics on the ocean has gained much
traction with scientists and politicians alike. Microplastics consist of
tiny particles of plastic and other pollution and are therefore are
difficult to detect. Currently there is no hydrodynamic model that is
able to determine dispersion, source and accumulation of microplastics
along the South African coastline. Given the significant detrimental
effect microplastics play in the ocean, it is important that such a
mechanism be developed that will aid in better management of marine
pollution in South Africa. Furthermore, this project allows for a
multidisciplinary approach to be taken by combining ecology and coastal
oceanography.

Abiotic and biotic factors interact in complex ways which indirectly
determine behavioral and ecophysiological responses in organisms. For
example, when storms or strong currents form in sub-tidal habitats, sea
urchins form aggregations in order to reduce overall drag to avoid being
swept away by currents. Organisms that are motile my migrate into more
environmentally suitable areas when conditions become unfavorable or
food sources become depleted. In changing environments migration may
also allow organisms to extend their distributions. For example, ocean
warming off the coast of Western Australia has allowed tropical fish
species to extend their distribution into areas that were previously
characterised as temperate reefs. Sessile organisms are unable to
migrate into more environmentally suitable areas and are forced to
either adapt or suffer expiration. Sessile organisms may respond to
changing environmental conditions through changes in physiology. For
example, plants may produce heat shock proteins that help buffer the
effect of temperature increases. Sessile organisms may also adapt their
morphology,in order to persist in changing and harsh environments, such
as seaweeds.

Temperature and wave exposure have been shown to be important drivers of
seaweed distribution, physiological functioning, ability to recover,
population dynamics and morphology. Mechanical forces generated by the
hydrodynamic environment, in the form of sudden strong ocean currents or
storms, between 10- 20 m s-1 with accelerations of 400 m s-2 (Friedland
and Denny 1995) are the biggest threat to kelp survival. Kelps are able
to rapidly adapt their morphological characteristics to reduce drag and
avoid dislodgement (Blanchette 1997). For example a study by Koehl et
al.~(2008) showed that transplanted Nereocystis luetkeana plants from a
wave sheltered site to a wave exposed site changed their morphology to
flat blades and narrow laterals that are less prone to drag forces in
4-5 days. Another study by Fowler-Walker, Wernberg, and Connell (2006)
tested for differences in morphology of Ecklonia radiata between
wave-sheltered and wave-exposed sites and through a combination of in
situ sampling and transplantation of juvenile plants. The results showed
that morphology differed between wave-sheltered and wave-exposed sites
(thin thallus at sheltered sites and a narrow, thick thallus with a
thick stipe at exposed sites), and was consistent with previous studies.
Juveniles transplanted into wave exposed sites under went rapid
morphological adaptation, whilst the opposite was true for
wave-sheltered sites which showed slower morphological adaptation.

Kelp morphology may be distinct to a particular region with a specific
hydrodynamic environment and has the ability to raft far distances using
coastal currents, and may accumulate as beach-cast in areas far from its
original location. Therefore, kelp morphology may be used as a proxy for
determining its original location as well as aid in characterising
coastal currents. However, this approach must be combined with advances
in hydrodynamic modelling for a quantitative outcome.

Advances in numerical modelling has gained much traction in recent years
and has been applied in a variety of ways with regards to ecological
studies. For example, a study by Wang and Xia (2009) used the
Delft3D-Flow model to assess the hydraulic suitability of a stream as a
spawning ground for the Chinese Sturgeon (Acipenser sinensis) in the
Yangtze River. The authors calculated the horizontal mean velocity which
was used to assess the hydraulic environment of spawning ground. The
flow field state was determined through model simulation and
field-measured data used to validate the model. The results added to
existing scientific database for spawning ground hydraulic environmental
protection. Different numerical models can often be integrated to model
across ecosystem levels. For example a study by Leon et al.~(2003) used
integrated physical (Delft3D hydrological model) and bio-chemical
(Agricultural Non-point Source model) processes models to investigate
the possible impact on the Lake Malawi water quality due to management
actions performed at the watershed level.

Since wave energy is an important driver in marine ecosystems,
particularly kelp, the advances in hydrodynamic modelling offer a new
opportunity for multifactorial and quantitative approach to research in
marine ecosystems. The Delft3D and SWAN models have been used
successfully in previous studies regarding brine plume discharge,
impacts of storms, effects of climate change on the hydrological
environment etc. The models have not been designed for shallow
environments (\textless6m) and therefore may not be suitable to model
coastal hydrological environments. However these models may be adjusted
to suit coastal waters if they are combined with a new numerical model
which can be calibrated to suit these needs.

In recent years there has been growing attention on plastic pollution,
particularly in the ocean. Plastic pollution can be in the form of
macro- and microplastics. Microplastics are tiny plastic granules used
as scrubbers in cosmetics and air-blasting, and small plastic fragments
that originate from larger pieces of plastic known as macroplastics,
while macroplastics\ldots insert definition here\ldots{} The potential
harms of of plastic pollution in the marine environment was highlighted
in the 1970's and renewed interest has lead to research showing that
plastic pollution in the ocean are widespread. Plastics may become
bio-available to biota in the food-web which may cause problems with an
organism's physiological functioning. Furthermore, the relatively large
surface area and composition of microplastics provides an environment
that is able of adhering to organic pollutants. In other words
microplastics also act as a vector for transport and assimilation of
organic pollutants.

Therefore, this study not only enables research into the ecological
effects of the hydrological environment on an important habitat-forming
organism, it also offers the opportunity to improve on current
hydrological numerical models to suit coastal environments. This in turn
will allow investigation into the flow and accumulation of microplastics
which are regarded as a major threat to marine life. Furthermore, the
calibrated model could be applicable to other ecological studies such as
dispersal of benthic flora and fauna, climate change studies,
forecasting as some examples.

\hypertarget{kelp-environmental-drivers}{%
\section{Kelp environmental drivers}\label{kelp-environmental-drivers}}

Important environmental drivers of kelp individuals and communities
include light, substrata, salinity, sedimentation, nutrients,
temperature and wave exposure. Although studies have investigated the
effects of these important environmental drivers, the roles these
factors play is often difficult to evaluate as such factors may never be
fully independent of each other, i.e.~environmental factors are to some
extent are dependent on one another. Multifactorial studies have
attempted to explain combined effects, however these studies are often
limited to investigating combination of two or three environmental
drivers as inclusion of too many factors can lead to results that are
difficult to interpret. Environmental factors are highly variable on
temporal and spatial scales and their effects may also be dependent on
the life-stage of the organism, adding a further layer of complexity to
investigations.

Light is an important factor for kelp survival, however if light is
limited or excessive this may negatively impact kelp survival or growth.
Much of the past research into the role light plays into the functioning
of kelp (Bruhn and Gerard 1996; ???). For instance, solar ultraviolet
radiation has been shown to affect sub-canopy Ecklonia radiata
sporophytes when the canopy of mature Ecklonia radiata was removed (Wood
1987). The sub-canopy sporophytes experienced tissue damage,photopigment
destruction,reduced growth and decreased survivorship, thus inhibiting
their settlement and survival (Wood 1987). Laboratory experiments
revealed that the UV component of radiation, rather than intense
radiation itself, was responsible for the effects mentioned above. High
light stress has negative effects, such as photoinhibition and
photo-damage on Ecklonia cava sporophytes (Altamirano and Murakami
2004). Altamirano and Murakami (2004) found that Ecklonia cava is more
vulnerable to light stress conditions, and less likely to recover under
unfavourable conditions (Altamirano and Murakami 2004). Bolton and
Levitt (1985) showed that under sub-saturating irradiances and supra-
optimal temperatures Ecklonia maxima to showed a decrease in
reproductive rates and an increase in cell production. An additional
finding of this study was that despite the decrease in reproductive
rates, the final egg production per female was greater under these
conditions. The authors interpreted this an ecological adaptation that
may increase survival rates under times of stress or non - ideal
conditions (Bolton and Levitt 1985).

Depth does not affect kelp ecosystems directly, however a change in
depth causes fluctuations or changes in other environmental variables
such as water motion, light and temperature. Water motion also decreases
with depth, and some kelps better suited to deeper environments
(\emph{L. pallida}) replace those in the shallows (\emph{E. maxima})
(Dayton 1985; Gerard 1982). The increase in depth can lead to a decrease
in sunlight penetration, with some species better adapted for low-light
conditions than others, such as (\emph{L. pallida}). Temperature may
also change along a depth gradient due to a reduction in sunlight
penetration (Dayton 1985; Gerard 1982). Therefore depth does not
directly play a role in kelp functioning but may alter more influential
factors such as light and water motion.

The importance of nutrients in the functioning of kelps is well
understood (Dayton 1985; Gaylord, Nickols, and Jurgens 2012). Dissolved
nitrogen, and in particular nitrate, are important; however research has
also placed emphasis on phosphate and other trace compounds for
functioning of kelps (Dayton 1985). Additionally, some kelps have the
ability to store inorganic nitrogen in order to compensate for periods
of low nutrient availability, which has been observed for Laminaria and
Macrocystis (Dayton 1985; Gaylord, Nickols, and Jurgens 2012). Nutrient
stratification is also an important factor, particularly for canopy type
kelps. The concentration of nutrients at the surface is important to the
functioning and maintenance of the canopy. For instance kelp canopies in
California often deteriorate in the summer months when surface nitrate
levels are low (Jackson 1977). Water motion is important in the
assimilation of nutrients from the water column, and kelps have been
shown to adapt blade morphology in order to create more turbulence
around the boundary layer of the frond to enhance nutrient assimilation
(Wheeler 1980). Temperature has also been closely linked with nutrient
concentrations. Nutrients are often in higher concentrations in the
water column during low temperature events. This is often an indication
of an up-welling event, which brings cold and nutrient rich waters from
the bottom to the surface of the water column. Temperature can play a
direct role in the uptake of nutrients through effects on algal
metabolism; however this may vary from species to species (???).

Temperature is a driver of kelp species distributions and
ecophysiological processes, as well as a lesser role in morphological
adaptation\ldots example here\ldots The majority of kelp species are
arctic and temperate organisms, and the warming of ocean temperatures is
expected to cause a poleward biogeographical shift of species (Bolton et
al.~2012). There is evidence to suggest that South African kelp forests
are expanding due to ocean cooling (Bolton et al.~2012), possibly driven
by an intensification and increase in coastal upwelling (Blamey and
Branch 2012, Blamey et al.~(2015)). In South Africa there has been a
biogeographical shift eastward along the coast due to a change in
inshore temperature regime, making South Africa no exception to changing
ocean temperatures (Bolton et al.~2012). Macroalgae, such as kelps, can
react to an increase in surface temperatures in one of three ways: they
can migrate, adapt and die (Biskup et al.~2014). A study by Biskup et
al.~(2014) investigated the functional response of two kelp species
(Laminaria ochroleuca and Saccorhiza polyschides) to rising sea
temperatures. The functional responses of Saccorhiza polyschides was
measured for both the subtidal and intertidal habitats, to see what
affect non- optimal conditions (intertidal zone) had on the kelps (Rinde
and Sjøtun 2005). The study found that Laminaria ochroleuca exhibited a
poor ability to acclimatise and was dependent on the kelp’s life
history traits (Biskup et al.~2014). Therefore annual kelp species are
more likely to survive under non-ideal condition, and the intertidal
Saccorhiza polyschides, compared to the subtidal, showed a higher
physiological flexibility to changing conditions (Biskup et al.~2014).
This may be because the intertidal zone undergoes far more change than
the subtidal and therefore kelps in the intertidal are forced to adapt
to harsher conditions where fluctuations in temperature, sunlight,
turbidity and water motion are common. The effects on temperature have
also been investigated by Wernberg et al.~(2010). The study looked at
resilience of kelp beds along a latitudinal temperature gradient. Kelp
abundance is likely to decline with the predicted warming of ocean
waters Wernberg et al.~(2010) and although kelps have the ability to
acclimatize and adjust their metabolic performance, which in turn allows
them to change their physiological performance to mitigate the seasonal
fluctuations in temperature, this acclimatization is done at a cost
Wernberg et al.~(2010)\ldots link to paragraph on kelp
morphology\ldots{}

Other than temperature, wave exposure is also recognised as an important
driver of the marine environment, and macroalgae are no exception. Wave
exposure has been shown to play a role in determining distribution,
abundance, diversity, composition, growth (Cousens 1982) and
productivity (Pedersen and Nejrup 2012) of macroalgae communities. For
example, the width, vertical zonation and diversity of algal communities
often change predictably along gradients of wave exposure. Wave exposure
may also drive macroalgae communities indirectly through the alteration
in effect of another environmental driver. For instance, increasing
degrees of exposure may positively influence the amount of area
available to trap light on macroalgal fronds, as well as increasing
nutrient uptake due to increased turbulence in the boundary layer around
the frond (???). The most important direct effect of wave exposure on
macroalgal communities is through mechanical dislodgement, which
ultimately leads to expiration. Wave exposure is a complex abiotic
variable which varies spatially and temporarily in the marine
environment. Furthermore, the degree to which a macroalgae community is
exposed, is dependent on local site characteristics, such as bathymetry
and local wind patterns. Despite this fact, macroalgae have been able to
persist in often harsh and variable wave environments. Macroalgae are
sessile organisms and incapable of migrating when local conditions
become unsuitable. Therefore, macroalgae must adapt to the local wave
climate in order to persist and survive, and achieve this through
morphological adaptation. The morphology of macroalgae are not fixed
genetic traits. A study by Koehl et al.~(2008) showed that transplanted
Nereocystis luetkeana plants from a wave sheltered site to a wave
exposed site changed their morphology to flat blades and narrow laterals
that are less prone to drag forces in 4-5 days. Advances in genetic
techniques and taxonomy have revealed that species delineation based on
morphology has been inaccurate, and organisms that were once considered
two separate species are actually one species. For example, Moss (1948)
investigated the anatomy, chemical composition of Fucus spiralis at
three sites that varied in wave exposure (sheltered, medium exposure and
exposed). The authors found that individuals in exposed sites showed
less branching of thalli as well as variation physiological components,
such as organic nitrogen, mannitol, laminarian and alginic acid
concentrations. The authors also noted a `crumpling effect' displayed by
individuals from exposed sites and inferred that this strategy may
reduce overall drag. Other studies show that macroalgae in wave exposed
environments have morphologies that reduce overall drag, increase
strength of attachment or increase flexibility. There is also evidence
that morphological adaptation is driven by currents, and in fact may be
driving hydrological performance of macroalgae. Duggins et al.~(2003)
examined the direct and indirect flow effects on population dynamics,
morphology and biomechanics of several understorey macroalgae species.
These species included Costaria costata, Agarum fimbriatum, and
Laminaria complanata and \emph{Nereocystis luetkeana}. The results
showed that in wave impacted sites (wave exposed) had higher rates of
mortality, and no significance was found between survival of individuals
and tidal or current velocity. The authors concluded that although tidal
and current velocity did not play a significant role in determining kelp
survival, it did play a role in morphological adaptation. The results
from this study suggest that high current and tidal stresses are the
main driver of kelp morphological adaptation. This in turn make those
individuals more resilient to dislodgement to wave exposure. Wave
exposure is stochastic in nature compared to tidal and ocean currents
which are more regular in their frequency and magnitude. Therefore the
regular forces of tidal and ocean currents may make kelp individuals
more resilient to mechanical dislodgement over time.

The morphological adaptation that macroalgae display are driven by site
conditions, therefore individuals must be morphologically flexible to
persist in stochastic environments. This may be achieved through
different strategies and are species-specific, which can be directly
attributed the high diversity in morphological characters of algal
communities. For example some algae have fronds and others are
articulated coralline, and therefore these species would need to adapt
their morphology differently in order to persist. In general, flat
strap-like blades are common in areas that are exposed to high wave
energy, while at protected sites blade morphology is wide and undulated.

\hypertarget{the-mechanisms-of-morphological-adaptation}{%
\section{The mechanisms of morphological
adaptation}\label{the-mechanisms-of-morphological-adaptation}}

\hypertarget{waves-and-macroalgae-characteristics}{%
\subsection{Waves and macroalgae
characteristics}\label{waves-and-macroalgae-characteristics}}

In the wave swept nearshore one would expect organisms to reflect the
harsh hydrodynamic environment by being streamlined, small and amoured.
This is certainly the case for an array of fauna which often comprise of
hard, rigid bodies that are held firmly in place to the substratum such
as limpets and isopods\ldots\ldots{}

\hypertarget{ocean-and-coastal-waves}{%
\section{Ocean and coastal waves}\label{ocean-and-coastal-waves}}

\hypertarget{introduction}{%
\subsection{Introduction}\label{introduction}}

Regardless of the location around the world, waves are a feature of any
coastline and the marine environment, and are important manifestations
of energy in the ocean. Waves are not the movement of water particles
but rather the movement and propagation of energy through the ocean. The
source of this energy can be formed locally, such as wind-driven waves,
and or from distant locations in the ocean such as storms, know as
\emph{swell}. This energy is transferred from deeper water into
shallower water where it plays a role in driving complex marine
ecosystems as well as shaping the environments that they live in. Energy
that is left-over after these processes is transferred into heat energy,
and heats up the sand and rocks on which each wave hits. This energy can
also be harnessed in simple ways such as a surfer catching a wave, or
more complex ways such as capturing energy from the ocean environment
for electricity production. However, waves are stochastic in nature and
therefore the energy that propagates through the ocean is not always
consistent. Therefore, it is often difficult to quantify and predict
wave energy in the marine environment. Despite this fact, waves are
gaining more recognition for the role they play in shaping coastlines,
beaches and hence the communities and organisms that depend on these
systems.

Waves manifest themselves in different ways, which is also dependent on
the energy creating force, and can be classified into different
categories. For instance, ``chop'' are produced by local winds, while
``tsunamis''" are rare waves that are formed during a earthquakes or
landslides and can be produced from thousands of kilometers away. Waves
in the ocean are known as ``swell'' and are produced from far distant
storms, while the most consistent form of waves that interacts with the
coastlines, usually twice a day, are called ``tides''. Tides are
produced from differences in gravitational forces between the ocean,
earth's crust, sun and the moon. These classifications can be broken
down further and will be investigated later in this chapter. The focus
of this chapter will be understanding wave theory and how waves form and
propagate into shallow-coastal waters. This will be essential to later
chapters that aim to model wave energy and calibrate the produced model
with biotic variables.

\hypertarget{generating-and-restoring-forces}{%
\subsection{Generating and restoring
forces}\label{generating-and-restoring-forces}}

Waves are formed due to the constant interaction between two forces,
which are known as the ``generating force'' and the ``restoration
force''. The generating force is the force which pushes water from one
layer up into the other. Layers in the ocean are formed through
differences in temperature (thermocline) and salinity (pycnocline) which
creates a boundry for wave energy to move along. In other words, waves
occur along the boundaries formed in the ocean by various abiotic
processes, and therefore density boundaries are essential to the
propagation of energy through the ocean.

The generating force and restoring force occur along these boundaries,
the generating force pushes water up across ocean boundaries while the
restoring force pulls it back to where the boundary was originally,
trying to restore the balance in energy. This ``tug of war'' between the
two forces creates an oscillating motion between the boundary layers
which acts as a point of disturbance, sending out energy in all
directions. The disturbance energy will continue to manifest itself
provided the tug and pull between the two forces is occurring. Once the
generating force stops, and ultimately the energy from the point of
disturbance, the waves dissipate and water restores to its original
state. A simple example would be blowing on water in a cup. The blowing
of air onto the surface of the water in the cup creates a point of
disturbance that pushes the air boundary into the water boundary. The
restoring force is the surface tension of the water that is maintained
through hydrogen bonds between water molecules. Once blowing on the
water in the cup has stopped, the ripples or ``waves'' in the cup begin
to diminish in size until the surface tension returns to its original
state. The hydrogen bonds between the water molecules are stronger than
the force of gravity, and therefore the force of the surface tension
returns the water to its original state. These waves are known as
``capillary waves'' and is essentially residual energy after the
generating source has stopped. The restoring force can also be in the
form of gravity and are known as ``gravity waves''. Using the same
example, if one blows too hard, the water boundary is pushed up into the
air boundary, breaking the hydrogen bonds between water molecules which
allows gravity to return the water to its original state. All waves in
the ocean are either capillary waves or gravity waves, and their
classification will be dependent on the restoring force involved.

In nature, there are three kinds of generating forces. These are wind,
displacement of large volumes of water and uneven forces of
gravitational attraction between the Earth, Moon and Sun. Different
generating forces are associated with different wave heights, periods
and the type of wave produced insert appropriate figure. Wind-generated
waves comprise of capillary waves, chop, swell and seiche. Some seiche
can form from landslides and earthquakes but most of these waves are
known as a tsunami. Swell create the waves with the large heights (up to
100m), while tides can create the tallest.

\hypertarget{wave-physics-and-scales}{%
\subsection{Wave physics and scales}\label{wave-physics-and-scales}}

Waves have a number of characteristics which are depicted in figure ??,
and is an idealised representation of what a wave is. The amplitude is
the vertical distance from its midline or equilibrium surface or still
water level to its highest point known as the crest. The equilibrium
surface is the level the ocean would be if there were no waves, for a
wave to form a disturbance must occur below or above this line. The
trough is the same distance of the amplitude, however the measurement is
taken from the equilibrium surface to the lowest point. Wave height is
the vertical distance from crest to trough, and is equal to twice the
amplitude. The wavelength is the horizontal distance from a crest/trough
to the next crest/trough respectively. It is important to note that the
energy propagating from a disturbance will not reach the ocean floor in
deep-water environments. The depth below a wave where the water, and
anything in the water, feels no motion or disturbance is known as the
wave base. The wave base is calculated by descending vertically from the
equilibrium surface by a value equal to halve the wavelength. The water
particles in the ocean move in a circular orbit and hence return to
their original position. This is because waves in the ocean represent
moving energy, and not moving water.

\hypertarget{types-of-waves}{%
\subsection{Types of waves}\label{types-of-waves}}

There are a variety of waves that form in the ocean and all differ in
terms of period or wavelength (??image??). The longest wave that can
form in the ocean are known as trans-tidal waves, and are generated by
fluctuations in magnetism between the Earth's crust and atmosphere. The
magnetic push and pull between the Moon and the Sun creates waves with a
slightly shorter wavelength, known as tides. Their period and wavelength
can also range from a few hours to more than a day, and from a few
hundred to a thousand kilometers respectively. Storm surge tend to have
a slightly shorter wavelength and period compared to tides. When low
atmospheric pressure systems and high wind speeds in a storm it elevates
the ocean surface, generating storm surge, which may cause flooding in
coastal areas as it approaches the coastline. Tsunami's are on the lower
end of the scale, and are generated by earthquakes or submarine
`landslides. Their random nature makes them difficult to predict and
increase amplitude as they approach the coast. This can make them
considerably large along coastal areas and often cause immense damage
and loss of life.

\hypertarget{measuring-waves}{%
\subsection{Measuring waves}\label{measuring-waves}}

Waves are often thought of as an elevation of the sea surface from a
specific point over a period of time but this is obviously not he case.
This is known as \emph{surface elevation} and is the instantaneous
elevation of the sea surface above a specific point in a time record
(see figure \ref{fig:surface elevation figures}). Although surface
elevation does not represent a wave, it can be used to create a wave
profile. This is achieved by profiling the the surface elevation between
two successive \emph{downward zero-crossings} or \emph{upwardward
zero-crossings} (see figure \ref{fig:surface elevation figures}). Both
zero-crossings are symmetrical and essentially the same statistically.
However, in practice the downward zero-crossings are preferred as it
takes steepness of a approaching wave into account (the front, see
figure \ref{fig:surface elevation figures}) which is relevant to
characterising breaking waves. It is important to note that surface
elevation can be negative while a wave profile cannot.

\begin{figure}

{\centering \includegraphics{thesis_chapter_1_files/figure-latex/surface elevation figures-1} 

}

\caption{\label{fig:surface elevation figures}The definition of a wave in a time record of the surface elevation with downward zero-crossings (upper panel) and upward zero-crossings (lower panel).}\label{fig:surface elevation figures}
\end{figure}

Characterising waves in a wave record requires compromise both in the
statistical sense and the practical, as it is a balance between keeping
the record short enough to remain stationary and long enough for a
reasonable averages to be calculated. The waves are characterised in
terms of wave heights and wave periods for individual waves in the
record and then averaged over that specific time. For example, a time
record of 15-20 minutes is standard when calculating a wave profile.

\begin{figure}

{\centering \includegraphics{thesis_chapter_1_files/figure-latex/Hs_Tp_surface_elevation figure-1} 

}

\caption{The definition of wave height and wave period in a time record of the surface elevation (the wave is defined with downwrd zero-crossings.}\label{fig:Hs_Tp_surface_elevation figure}
\end{figure}

Waves are complex and various approaches, techniques and devices have
been developed in order to measure waves effectively. One of the ways
that has been used extensively used in the past but less so today, are
visual estimates which can be used to characterise \emph{significant
swell height} (Hs) and \emph{significant swell period} (Tp). Visual
estimates use 15 - 20 of the most well defined, higher waves of a number
of wave groups to characterise Hs and Tp. Although these parameters are
useful, they do not accurately reflect the waves that are occurring in
nature. Ocean waves are a combination of wind sea (short, irregular,
locally generated waves) and swell (long, smooth waves, generated by
distant storms) and so more parameters are needed to separate Hs and Tp
driven by wind sea and swell, i.e.~separate Hs and Tp parameters for
wind sea and swell. However, even with separate parameters for the
different types of waves would still not be enough to effectively
characterise waves in the ocean. In order to characterise the detail and
complexity of ocean waves a different approach must be used, know as the
\emph{spectral} approach. This approach is based on the idea that the
sea surface can be characterised as the summation of a large number of
harmonic wave components, calculated by the
\emph{random-phase/amplitude} model. The random-phase amplitude model is
a summation of wave components on a discrete scale, however a continuous
scale is more relevant to characterising and measuring of waves in the
ocean. The discrete spectrums are converted to continuous spectrums
through various arithmetic methods. Therefore, the random-phase
amplitude model is used to obtain a \emph{continuous variance density
spectrum} which is a statistical relevant measure for scientists and
engineers.

The wave spectrum is based on several spectra, namely the amplitude
spectrum, variance spectrum, variance density spectrum (discontinuous)
and the variance density spectrum (continuous). A wave spectrum can be
used to describe the sea surface as a stochastic process, however two
conditions (assumptions) must first be met 1) the observation must be
stationary and 2) the surface elevations must be Gaussian distributed.
As mentioned previously, the random-phase/amplitude model characterises
waves as harmonic and the time-record can be reproduced by the sum of a
large number of harmonic waves using a \emph{Fourier analysis}. A
Fourier analysis allows values of the amplitude and phase for each
frequency to be characterised, which in turn provides the amplitude and
phase spectrum for a single time-record. The amplitude spectrum is
calculated by averaging many time-records which determines the
\emph{average amplitude spectrum}. This approach allows the entire wave
record to be characterised but is not statistically relevant and so is
not applicable to hydrodynamic modelling. In order for to attain a
statistically relevant value, the variance of each wave component must
be calculated. The variance is regarded as statistically relevant, as
the sum of the variances (i.e.~random surface elevation) of the wave
components is equal to the variance of the sum of the wave components.
In addition, LWT dictates that the energy of waves is proportional to
the variance and is therefore an indirect measure of wave energy. This
indirect measurement can also be used to determine other important wave
components such as wave-induced particle velocity and pressure
variations. Important to note that the measurement of variance are
discrete and is therefore limited in characterising waves. A continuous
scale is more applicable for hydrodynamic modelling purposes and is also
closer to representing ocean waves and can be calculated by allowing the
frequency interval () to approach zero.

\hypertarget{currents}{%
\section{Currents}\label{currents}}

\hypertarget{ocean-currents}{%
\subsection{Ocean currents}\label{ocean-currents}}

\hypertarget{nearshore-currents}{%
\subsection{Nearshore currents}\label{nearshore-currents}}

\hypertarget{ocean-modelling}{%
\section{Ocean modelling}\label{ocean-modelling}}

An ocean model is essentially a representation of physical processes in
the ocean in the form of equations or computer code which aids in
furthering our understanding of how the ocean works. Examples of
physical processes include the exchange of energy, mass, and momentum
between the ocean and external drivers; ocean movement/dynamics; and 3D
mixing and dissipation processes. Examples of process which exchange
energy, mass, and momentum between sources include, but are not limited
to, radiation, evaporation, precipitation, river runoff and wind energy.
Ocean movement and dynamics, as the name suggests, refers to processes
which determine the movement of water in the ocean, such as horizontal
advection and vertical convection. The 3D mixing and dissipation refers
to processes that remove energy from the system such as turbulence
caused by temperature differences and wind forcing. There are several
approaches to modelling the ocean all of which have advantages and
limitations. Conceptual or process models are limited in the different
processes that can be incorporated into the model and are often regarded
as simple, low resolution models. However conceptual or process models
make it easier to interpret dynamics and processes in isolation, and can
be run parallel to other models, and can accommodate larger simulations
compared to more complex models. In essence, conceptual or process
models are mathematical tools to aid in contributing to the theory of
ocean processes. Another type of approach are Earth System Models of
Intermediate Complexity (EMICs), which have a higher resolution and can
run for long timescales with relatively low computation cost. These
approaches have been used to investigate coupling of climate and abiotic
processes, such as ice sheets and carbon feedback loops. Although EMICs
incorporate more complexity, these approaches do not accurately reflect
natural ocean processes; instead General Circulation Models (GCMs)
provide a better description of the ocean. General circulation models
are capable of incorporating multiple processes at a higher resolution.
Generally speaking, currently GCMs provide the most accurate way of
describing natural ocean processes.

Ocean models are popular tools for investigating future climate
scenarios, but have a range of applications outside the realm of
academia. Ocean models can also be used in operational oceanography in
the maritime and shipping industry (now-casts and forecasts),
experimental oceanography, and can be used to mechanistically interpret
ocean observations. Ocean models are similar to atmospheric models but
fundamentally differ in some respects. For example, atmospheric models
involve air which is a compressible gas while ocean models involve
seawater which is nearly in-compressible. These differences have
significant impacts for how aspects of volume, temperature and pressure
influence one another. In air, the interplay between these factors is
largely linear and well understood, while in the ocean the relationship
is non-linear and poorly understood. For ocean models the equation state
used is \[\rho = f(x)TSP\] where, \(\rho\) represents density and is a
function of, \(T\) temperature, \(S\) salinity, and \(P\) pressure.
Although the equation state is more complex compared to that of air,
some aspects are slightly simpler as seawater is in-compressible and
therefore the water entering a grid box will be the same as the water
exiting a grid box. Another major difference between ocean and
atmospheric models is in ocean models salinity must be taken into
account, while with atmospheric models humidity is an important factor.
Ocean and atmospheric models also differ slighlty in terms of vertical
structure i.e.~layering. The surface of the ocean is where most heating
and coolong occurs which cause the formation of a marine boundry or
mixed layer. Below the mixed layer the ocean becomes more stratified
which is ideal for determining flow. However, complications arise in the
horizonal layer which does not continiously layer like the atmosphere.
The horizontal layer in ocean models needs to take into account
irregular bathymetry, shape and size of continental shelves/basins.
Furthermore, run-off from rivers can cause changes in the density of
seawater along the coasts, thereby altering their flow regimes.
Therefore, ocean models need to resolve both the vertical and horizontal
structuring, particulary along the ocean margins, which can be achieved
through selecting the correct horizontal and vertical coordinate
systems. Finally, the ocean interior is largely driven by density
gradients compared to the atmosphere which is well mixed and so the
equalibrium timescales are much slower relative to the atmosphere. For
instance, an ocean model can take as long as 1000 years to start from
rest.

Ocean models use similar equations to atmospheric models, with some
minor differences (might add equation list). Theses equations form the
basis of a ocean model, but more input is needed for a model to run
successfully. Ocean models also need information on boundary conditions
such as basin geometry, bathymetry and atmospheric pressure. The
Atmospheric pressure can be solved for dynamically, or at the very
least, the effects are approximated and imposed. Initial abiotic
conditions are another important input into ocean models. Initial
conditions include the mean state of temperature, salinity and velocity
fields and can be calculated from climatology data or from a previously
run ocean model. Finally, the forcing fields, both dynamic and static,
are needed such as shortwave radiation, long-wave radiation, latent
heat, evaporation, precipitation and land surface run-off. Examples of
dynamic forcing fields include winds and tides. One of the most
important steps in ocean modelling id defining the horizontal and
vertical grids to be used. In terms of horizontal grids, there are two
main types, namely regular and irregular grids. As the name suggests, a
regular grid contains cells of the same size and dimensions i.e the grid
cells are evenly spaced. When a regular grid is placed over a spherical
earth, the grid cells vary in size and dimension resulting in a
curvilinear grid. In addition, the grid lines converge at poles making
modelling in those regions almost impossible. A popular solution is to
use a tripolar grid laid over the Artic polar region which results in
two poles positioned over land, creating more even spacing at the
convergence points. The advantage of using regular grids is the decades
of research on the subject, relatively straight forward analysis
algorithms, and is computationally efficient. Problems arise with
regular grids when increasing the resolution over a specific region, as
the resolution must be increased along the entire latitude length. This
leads to modelling areas not of interest (middle of the ocean) and is a
waste of computational resources. As a result, the use of irregular
grids in ocean modelling has been growing in popularity in recent years,
as the spatial resolution can vary throughout the model.

There are different `types' of irregular grids available such as using
triangles meshed together to form a grid instead of rectangles or
squares and is known as `finite element modelling'. The size of the
triangles can be altered to construct a non-uniform horizontal
resolution over the computational domain. For example, the model
resolution can be finer for coastal regions and coarser resolution for
areas in the middle of the ocean that are not of interest. Therefore,
irregular grids can be customised according to what the model is
attempting to resolve. Furthermore, irregular grids can accuratley
represent areas/regions of complex coastal morphology, bathymetery and
physical complex nearshore systems. However, finite element modelling is
not without its disadvantages, one being that these models struggle to
resolve flow driven by geostrophy and advection, both of which are
important characteristics of large-scale ocean flow. Issues also arise
when modelling regions which vary in terms of viscosity and diffusivity
coefficients. In addition, the issues of numerical diffusion in ocean
models is compounded by spatially variable advection schemes when using
irregular grids, making these issues harder to resolve. In addition,
issues arise when trying to match regions of low resolution with regions
of high resolution which imposes limitations on how ocean models resolve
certain processes. However, irregular grids are common in coastal and
estuarine models which need a high resolution of a specific, spatially
complex area. Despite the limitations of using irregular grids for large
scale ocean modelling, some predict that the challenges will be overcome
and the use of regular grids for large scale modelling will become
common place. For example, the development of an irregular, adaptive
grid which dynamically changes the resolution as a function of flow in
time.

In terms of vertical grids, there are also different `types' that can be
used, all of which have advantages and limitations. The vertical grid is
sometimes reffered to as `vertical coordinate system', and several
aspects need to be taken into account when selecting a vertical grid
system. For example one needs to take into account that most of the
movement in the ocean is driven by movement at the surface, the ocean is
strongly stratisfied, flow occurs along density gradients, and complex
bathymetry over the computational domain. As mentioned previously, there
are variety of vertical coordinate systems to suite a variety of
applications with \(Z\), \(\theta\), and \(\rho\) vertical coordinate
systems regarded as the most popular. The absolute depth or
\(Z\)-coordinate system is based on a series of depth levels and is easy
to setup, is computationally efficient, and produces no pressure
gradient errors. In addition, the resolution can be increased for a
desired depth by simply reducing the spacing between the depth levels.
However, when increasing the resolution along lateral boundaries
(e.g.~continental slopes) additional grid cells are needed. This leads
to a similar problem as horizontal regular grids, where computational
resources are wasted on areas which do not need to be resolved. The
terrain following or \(\theta\)-coordinate system is based on
\emph{frictional} depth scaled from 0 to 1. For example, a
0.01\(\theta\) level is 1\% of the depth of the ocean, and
0.99\(\theta\) level represents the 99\% of the depth of the ocean. The
advantage of this approach is the bathymetery is accurately represented,
which allows high resolution near the seafloor, regardless of depth or
proximity to land. However, a \(\theta\)-coordinate system does not
resolve pressure gradients accurately as well as variable numerical
advection schemes. The last popular approach is the isopycnal or density
or \(\rho\)-coordinate system, which defines the grid based on density
layers. Density layers are important drivers of flow in the ocean
interior, which the \(\rho\)-coordinate system is able to characterise.
The assumption made for this approach is ocean currents generally flow
along surfaces of equal density for ocean interiors, meaning the flow is
adiabatic. This approach works well for ocean interiors but performs
poorly in shallow waters, where there is very little to no
stratification in the water column. Therefore, in order for this
approach to be applicable the fluid which flow is being resolved must be
stratified. Since each of the approaches differ in terms of advantages
and limitations, it is not uncommon to use a combination of coordinate
systems known as hybrid coordinate systems. The hybrid coordinate system
aims to optimise performance by combining the best suited coordinate
system for a particular region. An example of a model that uses a hybrid
coordinate system is the HYCOM model, which uses a combination of \(Z\)
coordinates for the surface layer and \(\rho\) density in the ocean
below. Hybrid coordinate systems allow vertical coordinates to evolve in
both temporally and spatially as the depth changes. Although a
dynamically optimised coordinate system improves the results of the
model, it is done at a high computational cost.

Another important component of an ocean model is the resolution, in
other words the size of the grid. Important to note is that the ocean
exhibits `multiple scale variability', which refers to the variability
in time and length scales of the ocean. For example, vertical turbulent
mixing occurs along a different time and length scale compared to the
formation of eddies and fronts. The range of time and space scales can
be from molecular (mm and seconds) to basin scale (10 000km and 1000
years). Furthermore, some processes are coupled across scales such that
large scale processes are coupled with small scale processes, and are
known as `non-local interactions'. These non-local processes can have
significant effects on both macro- and micro-scale components of the
ocean. The resolution of a model is generally defined by the ability to
permit or resolve eddies (see table below).

\begin{longtable}[]{@{}lll@{}}
\toprule
Resolution & Terminology & Definition\tabularnewline
\midrule
\endhead
\$\ge\(1\)\^{}\circ\$ & Coarse & No eddies\tabularnewline
\$\sim\(0.5\)\^{}\circ\$ & Eddy-permitting & Some eddies\tabularnewline
\$\le\(0.2\)\^{}\circ\$ & Eddy-resolving & Eddies generate at a
realistic strength and rate\tabularnewline
\bottomrule
\end{longtable}

Important to note is that the eddy `resolving-model' does not resolve
all eddies, or their assocaited aspects, but is a description of
resolving eddies on what is currently known so far.

The parametrisation of processes in a model is another important
component which must be accounted for to ensure computational resources
are not wasted on processes which are negligible or too complex. In
addition, the understanding of some processes include mesoscale eddy
effects, dense overflows, submesoscale eddy effects, coastal processes,
surface mixed layer process, friction, sub-grid scale mixing and
ocean-ice interactions. Finally, the model needs to be validated to
ensure it accurately represents a `natural' ocean. Validation can be
difficult as many observations are needed and the measurements
themselves can be biased depending on the instruments used. Validation
of ocean models has improved over the years due to initiatives such as
the Argo Float Program. Other examples include satellites which are
capable of measuring sea surface temperature, salinity and wind; while
\emph{in situ} instruments (buoys, ships, drifters, floats, temperature
recorders, weather stations) provide direct measurements for validation
purposes.

Ocean modelling is an important tool to help scientists understand how
ocean processes are effected by climate change. However, ocean models
are not perfect and challenges must be overcome to improve them, such as
the effects of the internal wave field, model bias, dynamic ice sheets,
mesoscale eddy fields, mixing and modelling drift. Furthermore, ocean
modelling has provided a basis and information for the development of
other tools such as hydrodynamic and trajectory modelling.

\hypertarget{hydrodynamic-modelling}{%
\section{Hydrodynamic modelling}\label{hydrodynamic-modelling}}

\hypertarget{introduction-1}{%
\subsection{Introduction}\label{introduction-1}}

Wave exposure may be modeled through various methods which range from
simple cartographic to more advanced numerical wave models. Traditional
ecological measures of wave exposure usually incorporates integrative
measures of hydrodynamic conditions at a particular site. Cartographic
models can be qualitative or quantitative and were designed for the need
of wave exposure measures to explain ecological distributions. A simple
set of calculations on coastline and wind data, and relatively small
input data sets are required. These are regarded as ``fetch-based
models'', which measure the length of open water associated with a site
along a straight line. The output of such an approach is a simplified
estimate of the potential wave energy for a specific set of sites.
Advances in cartographical methods using fetch-based models has allowed
for wave exposure measurements of larger areas, and has been suggested
as a method for predicting macroalgal community structure (???). An
example of such a model is the ``BioEx model'' which was developed by
Baardseth and others (1970) to estimate wave exposure over large
regions. BioEx requires frequency, strength and direction of winds,
weighted by degree of exposure within various directions. BioEx is
calculated as the sum of the index developed at different spatial scales
(local, fjord and open), and has been used in mapping marine coastal
biodiversity (???). Lindegarth and Gamfeldt (2005) criticised this
approach, arguing that the choice of wave exposure method can influence
ecological inference. The authors also highlighted the need for
objective, reproducible and quantitative studies comparing exposure
indices (Lindegarth and Gamfeldt 2005). Other authors, such as Hill et
al.~(2010), have argued that these simple measures can be improved upon
by including bathymetry data which allows the incorporation of
diffraction into the calculation. Diffraction is topographically induced
variations in wave direction. A model incorporating this complexity was
developed by Isæus (2004), and is known as the ``simplified wave model''
(SWM). The model uses measurements of wind strength, fetch and
empirically derived algorithms to mimic diffraction. Advances in
numerical modelling have been founded on physical wave theory which
describes how a wave ``behaves''. This approach is based on a
theoretical perspective rather than the need to answer ecological
questions. Besides diffraction, numerical models incorporate more
complexity by including wind forcing, wave-to-wave interactions and loss
of energy due to friction and wave breaking. Numerical models have a
variety of applications and are often incorporated within hydrodynamic
general circulation models and are used operationally for forecasting
the sea state (Group 1988; Booij, Ris, and Holthuijsen 1999; Smith,
Sherlock, and Resio 2001). The downside of advanced numerical models is
that are computationally intensive which creates limitations for large
scale simulations. Therefore, their application along long stretches of
variable coastline, inshore environments and ocean-wide simulations are
limited due to the poor spatial coverage. However, numerical models can
be designed for local or site specific coverage, provided the correct
data is available. Most models have been designed with a specific goal
or objective in mind, and so no single model exists that is appropriate
for all situations or applications. Although models are ``situation
specific'', completed models can set the foundation for future models
with regards to resolving important oceanographic processes, identifying
gaps, and identifying important tracer coefficients ({\textbf{???}}).
Modelling of nearshore ecosystems has been reviewed by Jones (2002), and
offers the reader a general overview of the discipline (\textbf{I must
try and get this paper}).

\hypertarget{practicalities-of-model-design}{%
\subsection{Practicalities of model
design}\label{practicalities-of-model-design}}

Hydrodynamic models use the same primitive equations for ocean models,
which calculate velocities, turbulence, temperature, and salinity, and
hence, density ({\textbf{???}}). The equations used in hydrodynamic
modelling need to be `discretised', which means the equations are
formulated to be evaluated at discrete temporal and spatial points.
Additionally, hydrodynamic models use similar approaches to grid and
coordinate structures (see ocean modelling section).

\hypertarget{delft-3d-numerical-suite}{%
\subsection{Delft-3D numerical suite}\label{delft-3d-numerical-suite}}

The Delft-3D numerical suite provides an advanced approach to
hydrodynamic modelling through consideration of various physical
phenomena and is a quantitative estimate of wave exposure. The suite of
models can be used for a range of applications such as simulating flow,
sediment transport, waves, water quality, coastal morphological
development and ecology. The suite of numerical models consists of two
modules; Delft-3D WAVE and Delft- 3D FLOW.

\hypertarget{delft-3d-wave}{%
\subsubsection{Delft-3D WAVE}\label{delft-3d-wave}}

The Delft-3D WAVE module uses the SWAN (Simulating Waves and Nearshore)
numerical models to simulate the generation and propagation of
wind-generated waves in coastal environments. The SWAN model is based on
discrete spectral action balanced equation and is fully spectral.
Spectral refers to the consideration of all wave directions and
frequencies and implies that short-crested random wave fields
propagating from different directions can be accounted for. The final
output of the model is wind-sea with superimposed swell.

\hypertarget{delft-3d-flow}{%
\subsubsection{Delft-3D FLOW}\label{delft-3d-flow}}

This module is a multi-dimensional hydrodynamic/transport simulation
program which calculates non-steady flow and transport processes that
result from tidal and meteorological forcing. The dimensions can be
either 2D or 3D and can be placed on a rectilinear or curvilinear,
boundary fitted grid.

\hypertarget{drifting-aspects-of-floating-objects}{%
\section{Drifting aspects of floating
objects}\label{drifting-aspects-of-floating-objects}}

The need to understand the effects of oceanographic conditions affecting
trajectory of floating objects has largely been borne from the maritime
industry. The applications in the maritime industry include locating
lost cargo, locating naval and plane debris, search and rescue, and the
hydrodynamic effects on naval architecture.

The trajectory of passively drifting objects on the sea surface is
influenced by multiple factors, such as water currents, atmospheric
wind, wave motion, wave induced currents, gravitational force and
buoyancy force. To complicate matters, the previously mentioned factors
do not act independently of one another but instead influence one
another. Furthermore, the gravitational and buoyancy forces on the
object are determined by the objects shape. Therefore, all these factors
need to be taken into account when modelling trajectory of drifting
objects. Given the local wind, surface current, and the shape and
buoyancy of the object is known, it is possible to estimate trajectory
by the equation \[V_{drift}=V_{curr} + V_{rel}\]

where \(V_{curr}\) represents the current velocity relative to the
earth, and \(V_{rel}\) represents the object drift velocity relative to
the ambient water. Ocean currents are determined by two components: the
surface current (including the effects of Ekman drift, baroclinic
motion, tidal and inertial currents) and Stokes drift induced by waves.
The assumption made is \(V_{curr}\) influences all floating objects in
the same manner, regardless of shape or size. Therefore, \(V_{curr}\) is
equated with the surface current obtained from a numerical ocean model
which has been parameterised by wind velocity. The effects of
\(V_{rel}\) on a floating object is driven by wind and wave forces which
is dependent on the shape and size of the floating object. Previous
research has shown that the effects of waves on drifting objects are
negligible when the length of the object is less than the wave length;
and increase significantly when the lengths are approximately the same
(Grue and Biberg 1993; Hodgins and Hodgins 1998). Furthermore, current
velocity \(V_{curr}\) affects on drifting objects is considered to be
negligible in this approach and ideally should be taken into account.
Studies investigating kelp rafts have identified surface current
velocity (as a function of wind velocity) as an important influence in
determine trajectory (Harrold and Lisin 1989; Macaya et al. 2005; Hobday
2000a, 2000b). Therefore, the objects motion/trajectory relative to the
wind must be taken into account.

\hypertarget{windage-factors}{%
\subsection{Windage factors}\label{windage-factors}}

An objects motion relative to the wind is sometimes referred to as
``windage factors'' or "leeway drift, and is difficult to accurately and
empirically describe. The difficulty is due to accurate measurements
needed for the current velocity, wind velocity, wave height and wave
direction. Furthermore, these factors do not act independently and
instead influence one another, which compounds the complexity of
calculating wind effects on drifting objects. To overcome this
challenge, measurements of current and wind velocity, and wave height
and direction are measured along the objects trajectory for a wide range
of conditions.

Wind effects or windage factors can be calculated using the approach by
Griffin, Oke, and Jones (2017), and can be expressed as
\[V = V_c + V_l + V_w\]

where \(V\) is the velocity of the drifting object, \(V_c\) is surface
current velocity, \(V_l\) is the `leeway' velocity and \(V_w\) is the
velocity due to wave forces. As stated, \(V_c\) is the velocity of the
surface current and is averaged over the vertical extend of the drifting
object. Velocity of the surface current is expressed as
\[V_c = V_5 + V_s\] where \(V_5\) is the average velocity for the the
top 5m of the ocean (estimated by an ocean model) and \(V_s\) is the
velocity effect of Stokes Drift due to wave action. It is important to
note that the approximation of Stokes drift does not take into account
finer-scale vertical differences of the velocity field, such as
Coriolis-Stokes forcing. Stokes drift generally manifests in
hydrodynamic environments with short-wavelengths and locally-generated
waves. Griffin, Oke, and Jones (2017) mitigated the exclusion of
finer-scale drivers of Stokes Drift by comparing modeled surface
velocities with those of \emph{in situ} drifters.

\(V_l\) is the velocity of the object relative to the water due to
effects of wind force directly on the object and can expressed by
\[V_l = w U_10\] where \(w\) is a linear `windage' or `leeway'
coefficient and \(U_10\) is the wind speed at 10m height. The study by
Griffin, Oke, and Jones (2017) used a 10m height to model drifting
airplane debris, however this may be adjusted according to drifting
object being modeled. The nature of \(w\) (linear windage) is dependent
on the object, for instance yachts tend to drift in a particular
orientation to the wind, and so the angle of the drifting object is
dependent on said orientation. Furthermore, drifting object which are
more wind exposed and are less submerged will have high windage. Due to
these factors, and to allow for a non-zero angle between \(V_l\) and
\(U_10\), \(w\) (linear windage) can be a complex number.

\(V_W\) is the velocity of the object as a result of direct wave force,
and is dependent on the size and shape of the object. In instances when
wave-wave interactions result in an small, unbalanced force then \(V_W\)
is assumed to be zero. Also, objects may be able to absorb wave energy
which can result in a non-zero \(V_W\) value. Since surface-waves are
wind driven, the \(V_W\) is calculated by \[V_W = f(U_10)\]

Important to note is that the effects of wind and waves on the
trajectory of a drifting object may be influenced by the object's
orientation. This has been noted in modelling drift trajectories of
ships (Allen and Plourde 1999; Hackett, Breivik, and Wettre 2006), and
should be incorporated into the leeway drift component. The direct
(\(V_l\)) and indirect (\(V_s\) and \(V_w\)) effects do not act
independently and instead influence one another. with regards to this,
Griffin, Oke, and Jones (2017) recommend incorporating an effective
windage factor which would represent the combined effects of Stokes
Drift, leeway drift and wave forces. Although a common approach, this
may not be necessary in all circumstances particularly if individual
effects of factors. As mentioned previously, the magnitude of effects of
waves and wind on a drifting object is dependent on its shape, size and
buoyancy.

\hypertarget{buoyancy-of-drifting-objects-in-the-oceans}{%
\subsection{Buoyancy of drifting objects in the
oceans}\label{buoyancy-of-drifting-objects-in-the-oceans}}

In order to determine the fraction of the kelp raft submerged the
\emph{Buoyancy Force} (BF) must be calculated. A theoretical example
using a object of a regular shape (see figure \ref{fig:BF_1}) will be
used to demonstrate the theory that will be applied to this component of
the study. First, the fraction of the object submerged can be
calculated, which in this case is represented by \(x\) (see figure
\ref{fig:BF_1}).

\begin{figure}

{\centering \includegraphics{thesis_chapter_1_files/figure-latex/BF_1-1} 

}

\caption{\label{fig:BF_1}Conceptual model of bouyancy characteristics.}\label{fig:BF_1}
\end{figure}

An object will only float if the object density (\(\rho\)) is less than
or equal to the density of the liquid in which the object is submerged,
i.e.~\(\rho_{object} < \rho_{liquid}\). The buoyancy force (\emph{BF})
is the force of the displaced liquid pushing the object in an upwards
direction which is equal to the weight of the displaced liquid and is
expressed as: \[BF = W_{displaced-liquid}\] The weight of the displaced
liquid can be calculated by multiplying the mass of the liquid by the
gravitational constant \(g\) and so the weight of the displaced liquid
can be calculated by: \[W_{displaced-liquid} = mg\] The equation is
expanded upon by including the calculation for mass, \(m = \rho V\), and
so now the weight of the displaced liquid can be calculated as:
\[mg = \rho_{liquid} V g\] where \(\rho\) represents the density of the
liquid and \(V\) represents the portion of the object submerged. Since
the object is floating, the BF can also be expressed by the weight of
the object and as mentioned previously the weight of the displaced
liquid can also be expressed by \(mg\), which in turn can be written as
\(W_{displaced-liquid} = W_{object}\). Now through substituting the
calculations for weight, volume and the dimensions of the object, the
equation can be simplified to:
\[x= h\displaystyle\frac{\rho_o}{\rho_l}\] where \(h\) is the overall
height of the object, \(\rho_o\) represents the density of the object
and \(\rho_l\) represents the density of the liquid.

It is important to note that this equation is only applicable to objects
that are floating i.e.~\(\rho_{object} < \rho_{liquid}\). The portion of
a floating object above the water surface can also be calculated using a
similar approach, and instead of calculating the submerged portion
(\(x\)) the calculation will now solve for the fraction above the
surface which is represented by \(y\) (see figure \ref{fig:BF_1}). The
same equation for calculating density
(\(\rho = \displaystyle\frac{m}{V}\)) will also be used in solving for
\(y\). As mentioned previously, for floating objects the
\(BF = m_{object}g\) and the BF can also be defined as the weight of the
displaced liquid (\(BF = W_{displaced-liquid}\)). Therefore, the weight
of the displaced liquid is equal to the weight of the object
(\(W_{displaced-liquid} = W_{object}\)). On this basis equation ?? can
be written as \[m_{liquid}g = m_{object}g\] Substituting the equation
for density into equation ?? the equation is now:
\[\rho_{o}V_{o}g = \rho_{l}V_{l}g\] Important to note is that \(V_{o}\)
represents the volume of displaced liquid i.e not the entire object
volume (see figure \ref{fig:BF_1}). The constant \(g\) is removed
through cancellation and now the equation can be written as:
\[\displaystyle\frac{V_l}{V_o} = \displaystyle\frac{\rho_o}{\rho_l}\] We
also now that the volume of the object submerged is equal to the volume
of the displaced liquid and the fraction
\(\displaystyle\frac{V_l}{V_o}\) represents the portion of the object
submerged. Therefore to calculate the portion above the surface this
must be subtracted by 1, therefore:
\[1-\displaystyle\frac{V_l}{V_o} = y\] and since
\(\displaystyle\frac{V_l}{V_o} = \displaystyle\frac{\rho_o}{\rho_l}\)
the equation can now be solved via density of the liquid and the object:
\[y = 1 - \displaystyle\frac{\rho_o}{\rho_l}\] Solving the equation
algebraically it can now be written
as:\[ y = \displaystyle\frac{\rho_l - \rho_o}{\rho_l}\]

Once the fractions of submerged and non-submerged portions of the object
have been calculated the next calculation relevant to this study would
be to calculate the weight needed to cause the object to sink or as
least be completely submerged. This will relate to epiphyte loading on
kelp rafts in order to determine raft longevity. If a weight is placed
on the object in the conceptual model that has enough weight to fully
submerge (but not sink) the object then an addition buoyancy force is
needed, which will be represented by \(BF_2\) which is equal to the
additional mass that has been added which can be expressed as \(mg\).
Based on the previous calculations and the laws of physics we know that:
\[W_{object} = mg = BF_2 = W_{add-liq-displaced}\] The weight of the
additional liquid displaced can now be calculated by substituting the
density and volume of the additional displaced liquid along with the
gravitational constant \(g\) which is expressed as:
\[W_{AL} = \rho_{l}V_{AL}g\] where the subscript \(AL\) represents the
components of \emph{additional liquid displaced}. The \(V\) can also
represent the change in volume (\(\Delta V\)) and so the expression can
be expanded to: \[W_{AL} = \rho(\Delta V) g\]\} where \(\Delta V\)
represents the change in volume of the object, i.e \(y\).

\hypertarget{buoyancy-factors}{%
\subsection{Buoyancy factors}\label{buoyancy-factors}}

The buoyancy of objects at sea has been shown to be influeced by
epibiont load (Hobday 2000b; Rothä usler et al. 2011; Graiff et al.
2016; Macaya et al. 2016), which reduces drift times by reducing
buoyancy with increasing biomass, which ultimatley leads to the object
sinking. The bouyancy of the drifting object is determined by the growth
rate of epiphytic species while drifting, which in turn will be depndent
on environmental factors such as temperature and light which will vary
along the objects trajectory Therefore, buoyancy should be parameterised
by temporal epibiont biomass load.

\hypertarget{particle-dispersion-modelling}{%
\section{Particle dispersion
modelling}\label{particle-dispersion-modelling}}

\hypertarget{particle-dispersion-models}{%
\subsection{Particle dispersion
models}\label{particle-dispersion-models}}

The Pol3DD model is able to track virtual particles to simulate water
borne dispersion of material. Examples include neutrally buoyant
anthropogenic material, larvae, oil spills, outfall discharges and
sediment transport (Lebreton, Greer, and Borrero 2012). The model uses a
second-order accurate advection scheme (Black and Gay 1990) described as
follows:
\[\delta\chi=\frac{({u'+ \frac{(u_y\upsilon'-\upsilon_yu')\delta t}{2} )\delta t}}{(1-\frac{u_x\delta t}{2})(1-\frac{\upsilon_y\delta t}{2})-\frac{u_y\upsilon_\chi\delta t^2}{4}}\]

\[\delta y=\frac{({\upsilon' + \frac{(\upsilon_\chi u'-u_\chi\upsilon) \delta t}{2})\delta t}}{(1-\frac{u_y\delta t}{2})(1-\frac{\upsilon_\chi\delta t)}{2}) - \frac{u_\chi\upsilon_y\delta t^2}{4}}\]
where

\[u'= u + \frac{\delta tu_t}{2}\]
\[\upsilon'=\upsilon + \frac{\delta t\upsilon_t}{2}\] and \(u\),
\(\upsilon\) are orthogonal velocity components, \(t\) is time,
\(\delta t\) is the model time step, \(u_\chi\), \(u_y\), \(\upsilon_y\)
are the \(u\) and \(\upsilon\) spatial velocity gradients, and \(u_t\),
\(\upsilon_t\) are temporal gradients. Horizontal diffusion was modeled
as a random walk with separate longitudinal and lateral coefficients set
to simulate random turbulence. The distance increments moved by the
particle at each time step is calculated as:
\[\Delta X = R_N \sqrt{6E_1\delta t}\]
\[\Delta y = R_N\sqrt{6E_2\delta t}\] where \(R_N\) is a random number
in the uniform range (-1,1) and \(E_1,E_2\) are the longitudinal and
lateral eddy diffusivities, respectively. To estimate the amount of
input into the model the authors used a scaled approach to define the
release of particles, and where based on previous research performed by
Halpern et al. (2008).

\hypertarget{application-of-particle-dispersion-models}{%
\section{Application of particle dispersion
models}\label{application-of-particle-dispersion-models}}

A study by Lebreton, Greer, and Borrero (2012) proposed a methodology to
track floating debris from source to sink based on descriptions of
global waste production and ocean surface currents, using a combination
of hydrodynamic and particle dispersion modelling. The particle tracking
model applied uses a two stage process, (1) a hydrodynamic model solves
the equations of motion to characterise the movement of water within the
domain of the model, (2) virtual particles are released into the flow
field established in the first step, and particles are allowed to move
through hydrodynamic forcing. The authors extracted sea surface currents
from the HYCOM/NCODA ocean circulation modelling system, which is forced
by the US Navy's Operational Global Atmospheric Prediction System
(NOGAPS). The model also took into account wind stress, wind speed, heat
flux, and precipitation and was looped five times to accurately
represent ocean circulation patterns for a 30 year period. The authors
used the velocity data aquired from HYCOM and coupled it to the
Lagrangian particle tracking model Pol3DD which was used to simulate
dispersion. No additional wind stress terms were applied to the motion
of particles, as this factor had already been incorporated in the HYCOM
hydrodynamic data. Potential extra stress for emerged parts of debris
was considered negligible as all floating debris are assumed to be fully
submerged. The Pol3DD model was also modified so provide additional
information on individual particles such as origin, age, and trajectory
information of individual particles. Since the model considered a
spherical earth, particles were allowed to re-enter the simulation if
they passed around the `globe'. Important to note is only the surface
layer was modelled in this study, as much of the evidence identifies
surface currents as the main dispersal vector of anthropogenic debris.

Griffin, Oke, and Jones (2017) added to the search for flight MH370 by
providing estimates of the trajectory of the debris. This was achieved
through a combination of an advanced ocean model calibrated with oceanic
drifters, and drifters in the form of aircraft parts. Unlike previous
work regarding the recovery of MH370 wreckage, this study incorporated
windage factors into the trajectory model as well as providing better
estimates of Stokes Drift.

\hypertarget{kelp-rafting}{%
\section{Kelp-rafting}\label{kelp-rafting}}

Dispersal is recognised as important driver of biodiversity, composition
and structure of ecological systems in the marine environment (Bernardes
Batista et al. 2018; {\textbf{???}}; Highsmith 1985; MacArthur and
Wilson 2001; Jackson and Sax 2010). These processes are largely
dependent on currents for dispersal of migrant populations which
ultimately promote connectivity of marine ecosystems (Bernardes Batista
et al. 2018; Mackas, Denman, and Abbott 1985, 1985; Zakas et al. 2009).
This is particularly true for organisms which lack pelagic larvae, and
are reliant on other modes of transport to reach new habitats through
passive modes of dispersal from anthropogenic and natural sources.
Anthropogenic sources include marine litter and structural debris while
examples of organic sources include pumice and macroalgae in the form of
kelp-rafts. Kelp-rafts have been identified as an important mode of
passive dispersal of marine organisms, such as invertebrates
{[}({\textbf{???}}); whichmann2012{]}, epiphytes {[}@{]}, grazers
(Nikula et al. 2010), invasive species (Lewis, Riddle, and Smith 2005)
and macroalgae themselves (Macaya et al. 2005) which either make
landfall or are transported offshore where they eventually sink. A
kelp-raft consists of an entanglement of macroalgae (one or multiple
species) which have been dislodged from the benthic environment through
hydrodynamic forces (mostly storms) and are positively buoyant by means
of gas-filled pneumatocysts and/or reproductive organs. Kelp-rafts are
capable of travelling vast distances {[}@{]} and are considered
important dispersal vectors in temperate latitudes, such as the Southern
California Bight (Hobday 2000a) and the Northern Baltic Sea {[}@{]}.

Kelp-raft abundance has been shown to vary temporally and spatially in
the ocean and around coastlines, which is dependent on seasonal growth
patterns. For example, studies investigating the dispersal patterns of
\emph{Sargassum} in the West Pacific show increased abundance during
growth seasons (spring and summer) as more biomass is available to
fragment (Deysher and Norton 1981; Kingsford 1992). Temporal variability
may also be related to seasonal storm frequency where more storms in a
particular season relate to an increase in kelp-raft abundance. For
example a study by Kingsford (1995) investigated the contribution of
\emph{Macrocystis pyrifera} rafts to habitat complexity in pelagic
environments. The authors found that the abundance of \emph{M. pyrifera}
rafts increased during seasons where storms occurred more frequently.

The trajectory of kelp rafts are determined by a combination of
oceanographic conditions and biological processes. Since kelp rafts
float, evidence points largely to wind driven surface currents as the
main driver, although the relative importance of wind versus surface
currents is largely unknown. For example a study by Harrold and Lisin
(1989) used radio-trackers on both natural and artificial
\emph{Macrocystis pyrifera} kelp rafts to investigate seasonal
trajectory patterns. The results showed that tracked kelp rafts were
mostly deposited nearby the source population and that seasonal wind
direction was the primary driver of trajectory for kelp rafts submerged
roughly \textasciitilde0.5 - 1m below the surface Harrold and Lisin
(1989). However, the authors did note the influence of surface currents
for some of their experiments when winds relaxed; which suggests that
surface currents may play a larger role in determining trajectory for
kelp rafts less exposed to wind or when prevailing winds are low.

\hypertarget{kelps-in-south-africa}{%
\section{Kelps in South Africa}\label{kelps-in-south-africa}}

The biogeographic distribution of kelp is limited by seawater
temperature (Bolton 2010), where increasing temperature gradients reduce
kelp distribution. Due to this limiting factor, the two main species of
kelps in southern African waters, \emph{Ecklonia maxima} and
\emph{Laminaria pallida}, are distributed along a section of the south
coast from De Hoop, extending west around the Cape Peninsula, and
thriving north into Namibia (Molloy and Bolton 1996, Stegenga et
al.~1997). This distribution follows a temperature gradient, where sea
temperatures increase as one moves south from Namibia, around Cape Point
and towards De Hoop. Although the two species occur together for the
majority of the coast, their basic morphologies and resource needs vary
to a degree. The larger species, \emph{E. maxima}, is distributed from
Lüderitz to Cape Agulhas (Fig. 1) (Bolton and Levitt 1985,Probyn and
McQuaid 1985, Bolton and Anderson 1987, Bolton et al.~2012). The
biogeographic distribution of kelp is limited by seawater temperature
(Bolton 2010), where increasing temperature gradients reduce kelp
distribution. Due to this limiting factor, the two main species of kelps
in southern African waters, Ecklonia maxima and* L. pallida*, are
distributed along a section of the south coast from De Hoop, extending
west around the Cape Peninsula, and thriving north into Namibia (Molloy
and Bolton 1996, Stegenga et al.~1997).

This distribution follows a temperature gradient, where sea temperatures
increase as one moves south from Namibia, around Cape Point and towards
De Hoop. Although the two species occur together for the majority of the
coast, their basic morphologies and resource needs vary to a degree. The
larger species, \emph{E. maxima}, is distributed from Lüderitz to Cape
Agulhas (Fig. 1) (Bolton and Levitt 1985, Probyn and McQuaid 1985,
Bolton and Anderson 1987, Bolton et al.~2012). Characterised by a large
distal swollen bulb filled with gas, and smooth fronds, this species
grows to approximately 10 meters (Bolton and Anderson 1987). There was,
however, a 17-meter specimen collected in 2015 off Cape Point (Smit,
unpubl. data).This species of kelp not only dominate the biomass of the
South African nearshore, but plays an important ecological role
(Bustamante and Branch 1996). The estimated productivity of \emph{E.
maxima} within South Africa varies between 350 and 1500g Cm-2yr-1 (Mann
1982). Across the majority of the coastline, Laminaria pallida remains a
subsurface kelp, dominating the kelp biomass at depths greater than 10
meters (Field et al.~1980a, Bolton and Anderson 1987, Molloy and Bolton
1996). This species is distributed from Danger Point, east of the Cape
Peninsula, to Rocky Point in northern Namibia, and reaches depths of
greater than 20 meters (Field et al.~1980a, Molloy and Bolton 1996,
Stegenga et al.~1997). Towards the north along the west coast, from
around Hondeklipbaai, \emph{L. pallida} replaces \emph{E. maxima} as the
dominant kelp species (Velimirov et al.~1977,Stegenga et al.~1997) and
it also occupies increasingly shallow subtidal regions. The northern
populations also exhibit an increase in stipe hollowness, compared to
the solid stipe morphs in the species' southern distributions (Molloy
and Bolton 1996). This variation in morphology was thought to represent
two distinct species, with the northern populations formerly described
as \emph{Laminaria schinzii} Foslie (Molloy and Bolton 1996). Genetic
work has subsequently shown that the two morphs are in fact the same
species (Rothman et al.~2017). In southern African waters, the primary
production of \emph{Laminaria pallida} is between 120 and 1900g C m2yr1,
similar to that of \emph{E. maxima} (Mann 1982). Primary production is
not the only pathway.

\hypertarget{aims-of-research}{%
\section{Aims of research}\label{aims-of-research}}

The aim of the project is to investigate coastal flow regimes along the
west coast and south-west coast of South Africa and the role this may
play in transport of kelp beach-cast and microplastics. This aim will be
met through the following objectives:

\begin{enumerate}
\def\labelenumi{\arabic{enumi}.}
\item
  Determine if the hydrodynamic environment is the main driver of kelp
  morphological characteristics using a numerical model
\item
  Simulate kelp rafting by means of particle dispersion modelling.
\item
  Conduct field experiments using artificial rafts and \emph{in situ}
  kelp to track the movement of kelp around the South African coastline
  using custom GPS trackers.
\item
  Use the experimental data to calibrate the model and investigate the
  role of storms in ocean dispersal patterns.
\end{enumerate}

\hypertarget{references}{%
\section*{References}\label{references}}
\addcontentsline{toc}{section}{References}

\hypertarget{refs}{}
\leavevmode\hypertarget{ref-allen1999review}{}%
Allen, A, and JV Plourde. 1999. ``Review of Leeway: Field Experiments
and Implementation. US Coast Guard Rep.'' CG-D-08-99, 351.

\leavevmode\hypertarget{ref-bernardes2018}{}%
Bernardes Batista, Manuela, Antônio Batista Anderson, Paola Franzan
Sanches, Paulo Simionatto Polito, Thiago Lima Silveira, Gabriela
Velez-Rubio, Fabrizio Scarabino, et al. 2018. ``Kelps' Long-Distance
Dispersal: Role of Ecological/Oceanographic Processes and Implications
to Marine Forest Conservation.'' \emph{Diversity} 10 (1): 11.

\leavevmode\hypertarget{ref-black1990}{}%
Black, Kerry P, and Stephen L Gay. 1990. ``A Numerical Scheme for
Determining Trajectories in Particle Models,'' 151--56.

\leavevmode\hypertarget{ref-Blamey2012}{}%
Blamey, Laura K, and George M Branch. 2012. ``Regime Shift of a
Kelp-Forest Benthic Community Induced by an `Invasion'of the Rock
Lobster Jasus Lalandii.'' \emph{Journal of Experimental Marine Biology
and Ecology} 420: 33--47.

\leavevmode\hypertarget{ref-Blamey2015}{}%
Blamey, L K, L J Shannon, J J Bolton, R Crawford, F Dufois, H
Evers-king, C Griffiths, et al. 2015. ``Ecosystem change in the southern
Benguela and the underlying processes.'' \emph{Journal of Marine
Systems} 144: 9--29.
\url{https://doi.org/10.1016/j.jmarsys.2014.11.006}.

\leavevmode\hypertarget{ref-Burrows2011}{}%
Burrows, M T, D S Schoeman, L B Buckley, P Moore, E S Poloczanska, K M
Brander, C Brown, et al. 2011. ``The pace of shifting climate in marine
and terrestrial ecosystems.'' \emph{Science} 334 (6056): 652--55.

\leavevmode\hypertarget{ref-Dayton1999}{}%
Dayton, P K, M J Tegner, P B Edwards, and K L Riser. 1999. ``Temporal
and spatial patterns of kelp demography: The role of Oceanographic
climate.'' \emph{Ecological Monographs} 69 (2): 219--50.
\url{https://doi.org/10.1890/0012-9615(1999)069\%5B0219:TASSOK\%5D2.0.CO;2}.

\leavevmode\hypertarget{ref-deysher1981}{}%
Deysher, Larry, and Trevor A Norton. 1981. ``Dispersal and Colonization
in Sargassum Muticum (Yendo) Fensholt.'' \emph{Journal of Experimental
Marine Biology and Ecology} 56 (2-3): 179--95.

\leavevmode\hypertarget{ref-doney2011}{}%
Doney, Scott C, Mary Ruckelshaus, J Emmett Duffy, James P Barry, Francis
Chan, Chad A English, Heather M Galindo, et al. 2011. ``Climate Change
Impacts on Marine Ecosystems.''

\leavevmode\hypertarget{ref-graiff2016}{}%
Graiff, Angelika, Jose F Pantoja, Fadia Tala, and Martin Thiel. 2016.
``Epibiont Load Causes Sinking of Viable Kelp Rafts: Seasonal Variation
in Floating Persistence of Giant Kelp Macrocystis Pyrifera.''
\emph{Marine Biology} 163 (9): 191.

\leavevmode\hypertarget{ref-griffin2017}{}%
Griffin, DA, PR Oke, and EM Jones. 2017. \emph{The Search for Mh370 and
Ocean Surface Drift}. Commonwealth Scientific; Industrial Research
Organisation.

\leavevmode\hypertarget{ref-grue1993}{}%
Grue, John, and Dag Biberg. 1993. ``Wave Forces on Marine Structures
with Small Speed in Water of Restricted Depth.'' \emph{Applied Ocean
Research} 15 (3): 121--35.

\leavevmode\hypertarget{ref-hackett2006}{}%
Hackett, Bruce, Øyvind Breivik, and Cecilie Wettre. 2006. ``Forecasting
the Drift of Objects and Substances in the Ocean,'' 507--23.

\leavevmode\hypertarget{ref-halpern2008}{}%
Halpern, Benjamin S, Shaun Walbridge, Kimberly A Selkoe, Carrie V
Kappel, Fiorenza Micheli, Caterina D'agrosa, John F Bruno, et al. 2008.
``A Global Map of Human Impact on Marine Ecosystems.'' \emph{Science}
319 (5865): 948--52.

\leavevmode\hypertarget{ref-Harley2012}{}%
Harley, C, K M Anderson, K W Demes, J P Jorve, R L Kordas, T Coyle, and
M H Graham. 2012. ``EFfects of Climate Change on Global Seaweed
Communities.'' \emph{Journal of Phycology} 48 (5): 1064--78.
\url{https://doi.org/10.1111/j.1529-8817.2012.01224.x}.

\leavevmode\hypertarget{ref-Harley2006}{}%
Harley, Christopher DG, A Randall Hughes, Kristin M Hultgren, Benjamin G
Miner, Cascade JB Sorte, Carol S Thornber, Laura F Rodriguez, Lars
Tomanek, and Susan L Williams. 2006. ``The Impacts of Climate Change in
Coastal Marine Systems.'' \emph{Ecology Letters} 9 (2): 228--41.

\leavevmode\hypertarget{ref-harrold1989}{}%
Harrold, Christopher, and Susan Lisin. 1989. ``Radio-Tracking Rafts of
Giant Kelp: Local Production and Regional Transport.'' \emph{Journal of
Experimental Marine Biology and Ecology} 130 (3): 237--51.

\leavevmode\hypertarget{ref-highsmith1985}{}%
Highsmith, Raymond C. 1985. ``Floating and Algal Rafting as Potential
Dispersal Mechanisms in Brooding Invertebrates.'' \emph{Marine Ecology
Progress Series. Oldendorf} 25 (2): 169--79.

\leavevmode\hypertarget{ref-hobday2000a}{}%
Hobday, Alistair J. 2000a. ``Abundance and Dispersal of Drifting Kelp
Macrocystis Pyrifera Rafts in the Southern California Bight.''
\emph{Marine Ecology Progress Series} 195: 101--16.

\leavevmode\hypertarget{ref-hobday2000b}{}%
---------. 2000b. ``Age of Drifting Macrocystis Pyrifera (L.) c. Agardh
Rafts in the Southern California Bight.'' \emph{Journal of Experimental
Marine Biology and Ecology} 253 (1): 97--114.

\leavevmode\hypertarget{ref-hodgins1998}{}%
Hodgins, Donald O, and Sandra LM Hodgins. 1998. \emph{Phase Ii Leeway
Dynamics Program: Development and Verification of a Mathematical Drift
Model for Liferafts and Small Boats}. Seaconsult Marine Research
Limited.

\leavevmode\hypertarget{ref-jackson2010}{}%
Jackson, Stephen T, and Dov F Sax. 2010. ``Balancing Biodiversity in a
Changing Environment: Extinction Debt, Immigration Credit and Species
Turnover.'' \emph{Trends in Ecology \& Evolution} 25 (3): 153--60.

\leavevmode\hypertarget{ref-Jennings2010}{}%
Jennings, Simon, and Keith Brander. 2010. ``Predicting the Effects of
Climate Change on Marine Communities and the Consequences for
Fisheries.'' \emph{Journal of Marine Systems} 79 (3-4): 418--26.

\leavevmode\hypertarget{ref-Johnson2011}{}%
Johnson, Craig R, Sam C Banks, Neville S Barrett, Fabienne Cazassus,
Piers K Dunstan, Graham J Edgar, Stewart D Frusher, et al. 2011.
``Climate Change Cascades: Shifts in Oceanography, Species' Ranges and
Subtidal Marine Community Dynamics in Eastern Tasmania.'' \emph{Journal
of Experimental Marine Biology and Ecology} 400 (1-2): 17--32.

\leavevmode\hypertarget{ref-Jones2002}{}%
Jones, JE. 2002. ``Coastal and Shelf-Sea Modelling in the European
Context,'' 45--48.

\leavevmode\hypertarget{ref-kingsford1992}{}%
Kingsford, Michael J. 1992. ``Drift Algae and Small Fish in Coastal
Waters of Northeastern New Zealand.'' \emph{Marine Ecology Progress
Series}, 41--55.

\leavevmode\hypertarget{ref-kingsford1995}{}%
---------. 1995. ``Drift Algae: A Contribution to Near-Shore Habitat
Complexity in the Pelagic Environment and an Attractant for Fish.''
\emph{Marine Ecology Progress Series. Oldendorf} 116 (1): 297--301.

\leavevmode\hypertarget{ref-Krumhansl2016}{}%
Krumhansl, Kira A, Daniel K Okamoto, Andrew Rassweiler, Mark Novak, John
J Bolton, Kyle C Cavanaugh, Sean D Connell, et al. 2016. ``Global
Patterns of Kelp Forest Change over the Past Half-Century.''
\emph{Proceedings of the National Academy of Sciences} 113 (48):
13785--90.

\leavevmode\hypertarget{ref-lebreton2012}{}%
Lebreton, LC-M, SD Greer, and Jose Carlos Borrero. 2012. ``Numerical
Modelling of Floating Debris in the World's Oceans.'' \emph{Marine
Pollution Bulletin} 64 (3): 653--61.

\leavevmode\hypertarget{ref-lewis2005}{}%
Lewis, Patrick N, Martin J Riddle, and Stephen DA Smith. 2005.
``Assisted Passage or Passive Drift: A Comparison of Alternative
Transport Mechanisms for Non-Indigenous Coastal Species into the
Southern Ocean.'' \emph{Antarctic Science} 17 (2): 183--91.

\leavevmode\hypertarget{ref-macarthur2001}{}%
MacArthur, Robert H, and Edward O Wilson. 2001. \emph{The Theory of
Island Biogeography}. Vol. 1. Princeton university press.

\leavevmode\hypertarget{ref-macaya2005}{}%
Macaya, Erasmo C, Sebastian Boltana, Ivan A Hinojosa, Juan E
Macchiavello, Nelson A Valdivia, Nelson R Vasquez, Alejandro H
Buschmann, Julio A Vasquez, JM Alonso Vega, and Martin Thiel. 2005.
``PRESENCE of Sporophylls in Floating Kelp Rafts of Macrocystis
Spp.(PHAEOPHYCEAE) Along the Chilean Pacific Coast 1.'' \emph{Journal of
Phycology} 41 (5): 913--22.

\leavevmode\hypertarget{ref-macaya2016}{}%
Macaya, Erasmo C, Boris López, Fadia Tala, Florence Tellier, and Martin
Thiel. 2016. ``Float and Raft: Role of Buoyant Seaweeds in the
Phylogeography and Genetic Structure of Non-Buoyant Associated Flora,''
97--130.

\leavevmode\hypertarget{ref-mackas1985}{}%
Mackas, David L, Kenneth L Denman, and Mark R Abbott. 1985. ``Plankton
Patchiness: Biology in the Physical Vernacular.'' \emph{Bulletin of
Marine Science} 37 (2): 652--74.

\leavevmode\hypertarget{ref-McGowan1998}{}%
McGowan, J A, D R Cayan, and Le Roy M Dorman. 1998. ``Climate-ocean
variability and ecosystem response in the Northeast Pacific.''
\emph{Science} 281 (5374): 210--17.
\url{https://doi.org/10.1126/science.281.5374.210}.

\leavevmode\hypertarget{ref-nikula2010}{}%
Nikula, Raisa, CI Fraser, HG Spencer, and JM Waters. 2010. ``Circumpolar
Dispersal by Rafting in Two Subantarctic Kelp-Dwelling Crustaceans.''
\emph{Marine Ecology Progress Series} 405: 221--30.

\leavevmode\hypertarget{ref-Poloczanska2013}{}%
Poloczanska, Elvira S, Christopher J Brown, William J Sydeman, Wolfgang
Kiessling, David S Schoeman, Pippa J Moore, Keith Brander, et al. 2013.
``Global Imprint of Climate Change on Marine Life.'' \emph{Nature
Climate Change} 3 (10): 919.

\leavevmode\hypertarget{ref-Polovina2005}{}%
Polovina, Jeffrey J. 2005. ``Climate Variation, Regime Shifts, and
Implications for Sustainable Fisheries.'' \emph{Bulletin of Marine
Science} 76 (2): 233--44.

\leavevmode\hypertarget{ref-rotha2011}{}%
Rothä usler, Eva, Iván Gómez, Iván A Hinojosa, Ulf Karsten, Leonardo
Miranda, Fadia Tala, and Martin Thiel. 2011. ``Kelp Rafts in the
Humboldt Current: Interplay of Abiotic and Biotic Factors Limit Their
Floating Persistence and Dispersal Potential.'' \emph{Limnology and
Oceanography} 56 (5): 1751--63.

\leavevmode\hypertarget{ref-zakas2009}{}%
Zakas, C, J Binford, SA Navarrete, and JP Wares. 2009. ``Restricted Gene
Flow in Chilean Barnacles Reflects an Oceanographic and Biogeographic
Transition Zone.'' \emph{Marine Ecology Progress Series} 394: 165--77.

\end{document}

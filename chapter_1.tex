\documentclass[]{article}
\usepackage{lmodern}
\usepackage{amssymb,amsmath}
\usepackage{ifxetex,ifluatex}
\usepackage{fixltx2e} % provides \textsubscript
\ifnum 0\ifxetex 1\fi\ifluatex 1\fi=0 % if pdftex
  \usepackage[T1]{fontenc}
  \usepackage[utf8]{inputenc}
\else % if luatex or xelatex
  \ifxetex
    \usepackage{mathspec}
  \else
    \usepackage{fontspec}
  \fi
  \defaultfontfeatures{Ligatures=TeX,Scale=MatchLowercase}
\fi
% use upquote if available, for straight quotes in verbatim environments
\IfFileExists{upquote.sty}{\usepackage{upquote}}{}
% use microtype if available
\IfFileExists{microtype.sty}{%
\usepackage{microtype}
\UseMicrotypeSet[protrusion]{basicmath} % disable protrusion for tt fonts
}{}
\usepackage[margin=1in]{geometry}
\usepackage{hyperref}
\hypersetup{unicode=true,
            pdftitle={Kelp morphometric properties and the link to hydrodynamic modelling},
            pdfauthor={Ross Coppin},
            pdfborder={0 0 0},
            breaklinks=true}
\urlstyle{same}  % don't use monospace font for urls
\usepackage{graphicx,grffile}
\makeatletter
\def\maxwidth{\ifdim\Gin@nat@width>\linewidth\linewidth\else\Gin@nat@width\fi}
\def\maxheight{\ifdim\Gin@nat@height>\textheight\textheight\else\Gin@nat@height\fi}
\makeatother
% Scale images if necessary, so that they will not overflow the page
% margins by default, and it is still possible to overwrite the defaults
% using explicit options in \includegraphics[width, height, ...]{}
\setkeys{Gin}{width=\maxwidth,height=\maxheight,keepaspectratio}
\IfFileExists{parskip.sty}{%
\usepackage{parskip}
}{% else
\setlength{\parindent}{0pt}
\setlength{\parskip}{6pt plus 2pt minus 1pt}
}
\setlength{\emergencystretch}{3em}  % prevent overfull lines
\providecommand{\tightlist}{%
  \setlength{\itemsep}{0pt}\setlength{\parskip}{0pt}}
\setcounter{secnumdepth}{5}
% Redefines (sub)paragraphs to behave more like sections
\ifx\paragraph\undefined\else
\let\oldparagraph\paragraph
\renewcommand{\paragraph}[1]{\oldparagraph{#1}\mbox{}}
\fi
\ifx\subparagraph\undefined\else
\let\oldsubparagraph\subparagraph
\renewcommand{\subparagraph}[1]{\oldsubparagraph{#1}\mbox{}}
\fi

%%% Use protect on footnotes to avoid problems with footnotes in titles
\let\rmarkdownfootnote\footnote%
\def\footnote{\protect\rmarkdownfootnote}

%%% Change title format to be more compact
\usepackage{titling}

% Create subtitle command for use in maketitle
\newcommand{\subtitle}[1]{
  \posttitle{
    \begin{center}\large#1\end{center}
    }
}

\setlength{\droptitle}{-2em}

  \title{Kelp morphometric properties and the link to hydrodynamic modelling}
    \pretitle{\vspace{\droptitle}\centering\huge}
  \posttitle{\par}
    \author{Ross Coppin}
    \preauthor{\centering\large\emph}
  \postauthor{\par}
      \predate{\centering\large\emph}
  \postdate{\par}
    \date{26 February 2017}


\begin{document}
\maketitle

{
\setcounter{tocdepth}{2}
\tableofcontents
}
\subsection{Summary}\label{summary}

Marine ecosystems are maintained by a variety of complex interactions
between abiotic and biotic variables such as temperature, wave exposure,
pH, competition, and processes such as top-down and bottom-up control,
predator-prey relationships and phenology {[}@Doney2012{]}. These
abiotic and biotic variables, the interactions between them, and the
various ecological processes, ultimately determine the community
composition and ecological functioning of all ecosystems. Climate
directly and indirectly affects the way in which abiotic and biotic
variables interact, but is often compounded by other impacts such as
habitat destruction, pollution, and over-fishing {[}@Blamey2015{]}.
Temperature and wave exposure have been recognised as important
variables with regards to climate-driven changes within the ocean. In
order to persist and survive within variable and changing environments,
organisms must either migrate, adapt, or die.

Seaweeds are sessile organisms which are unable to migrate to new areas
when local environmental conditions become unsuitable and therefore are
forced to adapt to new conditions in order to avoid expiration. The main
form of mortality for seaweeds is through mechanical dislodgment by wave
action. Seaweeds, particularly brown seaweeds, are able to undergo rapid
morphological adaptation to the hydrodynamic environment. This allows
seaweeds to reduce mortality through mechanical dislodgement by inducing
morphology which reduces overall drag. Seaweeds that are unable to avoid
mechanical dislodgement either raft out to sea or wash up onto beaches.
However, not all the beach-cast kelp may have originated from a nearby
kelp population and may have originated from other sites or regions of
the coast through rafting and ocean currents. Therefore, because kelp
morphology is specific to its local environment the morphological
features may be able to indicate, within a certain amount of
probability, what site or region it most likely originated from. In
other words, beach-cast kelp may be used as proxy for investigating the
flow of coastal currents. Using kelp as a proxy for determining its
original location will be calibrated by means of a hydrodynamic model
which will be designed from already existing SWAN and Delf3D models.
This combined approach will allow investigation into flow regimes around
the west and south-west coasts of South Africa and the role they play in
sub-tidal and beach ecology. For instance, micro-plastics are recognised
as a threat to marine life, however very little is known about how
micro-plastics may be transported along the coast.

The increased use of plastic in society over the past half century has
resulted in large amounts of plastic litter in both the marine and
terrestrial environment. The problems associated with large plastic
debris have received attention for many decades, whereas those connected
to marine micro-plastics comparatively received very little attention.
However, today it has become a prioritized area among political
organizations, agencies and NGOs around the world. The micro-plastic
debris present in the ocean are derived from marine and terrestrial
sources, however there is little understanding of how microplastics may
be distributed with ocean currents. Therefore, the coupling of kelp
morphology and the ability to simulate hydrodynamic processes can
greatly improve our understanding of transport pathways and likely
locations of accumulation. This in turn may inform management descions
with regards to elimating and managing marine pollution in South Africa.

\subsection{Background}\label{background}

Seaweeds, browns in particular, are capable of adapting morphological
characteristics to persist in changing and variable ocean environments.
Changes in morphology have been shown to increase photosynthetic
ability, enhance nutrient uptake, and reduce drag with regards to
changes in temperature and wave exposure. Wave exposure has been shown
to be an important driver of seaweed morphology, as the main mechanism
of seaweed mortality is through the dislodgment. Changing morphology
reduces drag and increases the probability of survival. However, locally
adapted seaweed may still be dislodged in pulse disturbance events such
as storms, and may raft far distances and wash up on beaches. An example
would be \emph{Ecklonia maxima} which is a conspicuous brown seaweed
along the coastline and beaches around South Africa. Therefore, the
beach-cast may not always originate from adjacent kelp populations but
rather from other regions which the individual is adapted for.
Therefore, the kelp morphology may act as a proxy for investigating
coastal currents and changes thereof. The advances in ocean hydrodynamic
modelling has made great progress and has been applied in a variety of
ways. Therefore, this study will use advances in hydrodynamic modelling
in combination with kelp morphological characteristics to investigate
coastal currents. Once the model has been established it may be applied
in other ways, such as investigating the transport of micorplastics
along the South African coastline. The harms of microplastics to the
marine environment has gained much traction in recent years, but
research in South Africa is lacking. The coupling of kelp morphology and
the ability to simulate hydrodynamic processes can greatly improve our
understanding of transport pathways and likely locations of
accumulation. This in turn may inform management decisions with regards
to eliminating and managing marine pollution in South Africa.

Pollution is a huge environmental problem that affects both terrestrial
and marine ecosystems. Pollution from land enters the sea where it could
harm or kill marine organisms or is transported by ocean currents to
other coastal areas. In these areas the pollution could re-enter the
ocean or be blown by wind into terrestrail ecoystems where it could once
again be ingested by oragnsims causing harm or even death. Pollution is
therefore a major threat to the environment and marine organisms. In
recent years the affect of microplastics on the ocean has gained much
traction with scientists and politictions alike. Microplastics cosiste
of tiny particles of plastic and other pollution and are thereofore are
difficult to detect. Currently there is no hydrodynamic model that is
able to determine dispersion, source and accumulation of microplastics
along the South African coastline. Given the significant deterimental
affect microplastics play in the ocean, it is important that such a
mechanism be developed that will aid in better management of marine
pollution in South Africa. Furthermore, this project allows for a
multidicisplinery approach to be taken by combining ecology and coastal
oceanography.

Abiotic and biotic factors interact in complex ways which indirectly
determine behavioral and ecophysiological responses in organisms. For
example, when storms or strong currents form in sub-tidal habitats, sea
urchins form aggregations in order to reduce overall drag to avoid being
swept away by currents. Organisms that are motile my migrate into more
environmentally suitable areas when conditions become unfavorable or
food sources become depleted. In changing environments migration may
also allow organisms to extend their distributions. For example, ocean
warming off the coast of Western Australia has allowed tropical fish
species to extend their distribution into areas that were previously
characterised as temperate reefs. Sessile organisms are unable to
migrate into more environmentally suitable areas and are forced to
either adapt or suffer expiration. Sessile organisms may respond to
changing environmental conditions through changes in physiology. For
example, plants may produce heat shock proteins that help buffer the
affect of temperature increases. Sessile organisms may also adapt their
morphology,in order to persist in changing and harsh environments, such
as seaweeds.

Temperature and wave exposure have been shown to be important drivers of
seaweed distribution, physiological functioning, ability to recover,
population dynamics and morphology. Mechanical forces generated by the
hydrodynamic environment, in the form of sudden strong ocean currents or
storms, between 10- 20 m s-1 with accelerations of 400 m s-2
{[}@Friedland1995{]} are the biggest threat to kelp survival. Kelps are
able to rapidly adapt their morphological characteristics to reduce drag
and avoid dislodgment {[}@Blanchette1997{]}. For example a study by
@Koehletal2008 showed that transplanted \emph{Nereocystis luetkeana}
plants from a wave sheltered site to a wave exposed site changed their
morphology to flat blades and narrow laterals that are less prone to
drag forces in 4-5 days. Another study by @FowlerWalker2006 tested for
differences in morphology of \emph{Ecklonia radiata} between
wave-sheltered and wave-exposed sites and was a combination of \emph{in
situ} sampling and transplant of juvenile plants. The results showed
that morphology differed between wave-sheltered and wave-exposed sites
(thin thallus at sheltered sites and a narrow, thick thallus with a
thick stipe at exposed sites), and was consistent with previous studies.
Juveniles transplanted into wave exposed sites under went rapid
morphological adaption, whilst the opposite was true for wave-sheltered
sites which showed slower morphological adaption.

Kelp morphology may be distinct to a particular region with a specific
hydrodynamic environment and has the ability to raft far distances using
coastal currents, and may accumulate as beach-cast in areas far from its
original location. Therefore, kelp morphology may be used as a proxy for
determining its original location as well as aid in characterising
coastal currents. However, this approach must be combined with advances
in hydrodynamic modelling for a quantitative outcome.

Advances in numerical modelling has gained much traction in recent years
and has been applied in a variety of ways with regards to ecological
studies. For example, a study by @wang2009 used the Delft3D-Flow model
to assess the hydraulic suitability of a stream as a spawing ground for
the Chinese Sturgeon (\emph{Acipenser sinensis}) in the Yangtze River.
The authors calculated the horizontal mean vorticity which was used to
assess the hydraulic environment of spawning ground. The flow field
state was determined through model simulation and field-measured data
used to validate the model. The results added to existing scientific
database for spawning ground hydraulic environmental protection.
Different numerical models can often be integrated to model across
ecosystem levels. For example a study by @leon2003 used integrated
physical (Delft3D hydrological model) and bio-chemical (Agricultural
Non-point Source model) processes models to investigate the possible
impact on the Lake Malawi water quality due to management actions
performed at the watershed level.

Since wave energy is an important driver in marine ecosystems,
particularly kelp, the advances in hydrodynamic modelling offer a new
opportunity for multifactoral and quantitative approach to research in
marine ecosystems. The Delf3D and SWAN models have been used
successfully in previous studies regarding brine plume discharge,
impacts of storms, affects of climate change on the hydrological
environment etc. The models have not been designed for shallow
environments (\textless{}6m) and therefore may not be suitable to model
coastal hydrological environments. However these models may be adjusted
to suit coastal waters if they are combined with a new numerical model
which can be calibrated to suit these needs.

In recent years there has been growing attention on plastic pollution,
particulary in the ocean. Plastic pollution can be in the form of macro-
and microplastics. Microplastics are tiny plastic granuales used as
scrubbers in cosmetics and air-blasting, and small plastic fragments
that originate from larger pieces of plastic known as macroplastics,
while macroplastics\ldots{}\textbf{insert definition here}\ldots{} The
potential harms of of plastic pollution in the marine environment was
highlighted in the 1970's and renewed interest has lead to research
showing that plastic pollution in the ocean are widespread. Plastics may
become bio-available to biota in the food-web which may cause problems
with an organisms physiological functioning. Furthermore, the relatively
large surface area and composition of microplastics provides an
environmeht that is able of adhering to organic pollutants. In other
words microplastics also act as a vector for transport and assimilation
of organic polluants.

Therefore, this study not only enables research into the ecological
affects of the hydrological environment on an important habitat-forming
organism, it also offers the opportunity to improve on current
hydrological numerical models to suit coastal environments. This in turn
will allow investigation into the flow and accumulation of microplastics
which are regarded as a major threat to marine life. Furthermore, the
calibrated model could be applicable to other ecological studies such as
dispersal of benthic flora and fauna, climate change studies,
forecasting etc.

\section{Kelp environmental drivers}\label{kelp-environmental-drivers}

The important environmental drivers of kelp individuals and communities
include light, substrata, salinity, sedimentation, nitrients,
temperature and wave exposure. Although studies have investigated the
effects of important environmental drivers, the roles these factors play
is often difficult to evaluate as such factors may never be fully
independent of each other, i.e.~environmental factors are to some extent
dependent on one another. Multifactorial studies have attempted to
explain combined affects, however these studies are often limited to
investigating combination of two or three environmental drivers as
inclusion of too many factors can lead to results that are difficult to
interept. Environmetal factors are highly variable on temporal and
spatial scales and their effects may also be dependent on the life-stage
of the organism, adding a further layer of complexity to investigations.

\section{Light}\label{light}

Light is an important factor for kelp survival, however if light is
limited or excessive this may negatively impact kelp survival or growth.
Much of the past research into the role light plays into the functioning
of kelp {[}@Bruhn1996);@Belseth2012{]}. For instance, solar ultraviolate
radiation has been shown to affect sub-canopy Ecklonia radiata
sporophytes when the canopy of mature \emph{Ecklonia radiata} was
removed {[}@Wood 1987{]}. The sub-canopy sporophytes experienced tissue
damage,photopigment destruction,reduced growthand decreased
survivorship, thus inhibiting their settlement and survival {[}@Wood
1987{]}. Laboratory experiments revealed that the UV component of
radiation, rather than intense radiation itself, was responsible for the
effects mentioned above. High light stress has negative effects, such as
photoinhibition and photo-damage on \emph{Ecklonia cava} sporophytes
{[}@Altamirano2004{]}. @Altamirano2004 found that \emph{Ecklonia cava}
is more vulnerable to light stress conditions, and less likely to
recover under unfavourable conditions {[}@Altamirano2004{]}. @Bolton1985
showed that under sub-saturating irradiances and supra- optimal
temperatures \emph{Ecklonia maxima} to showed a decrease in reproductive
rates and an increase in cell production. An additional finding of this
study was that despite the decrease in reproductive rates, the final egge
production per female was greater under these conditons. The authors
interpreted this an ecological adaption that may increase survival rates
under times of stress or non - ideal conditions {[}@Bolton1985{]}.

\section{Substrata}\label{substrata}

\section{Salinity}\label{salinity}

\section{Depth}\label{depth}

Depth does not affect kelp ecosystems directly, however a change in depth
often causes fluctuations or changes in other environmental variables
such as water motion, light and temperature. Water motion also decreases
with depth, and some kelps better suited to deeper environments
(\emph{L. pallida}) replace those in the shallows (\emph{E. maxima})
{[}@Dayton1985; @Gerard1982{]}. The increase in depth can lead to a
decrease in sunlight penetration, with some species better adapted for
low-light conditions than others, such as (\emph{L. pallida}).
Temperature may also change along a depth gradient due to a reduction in
sunlight penetration {[}@Dayton1985; @Gerard1982{]}. Therefore depth
does not directly play a role in kelp functioning but may alter more
influential factors such as light and water motion.

\section{Sedimentation}\label{sedimentation}

\section{Nutrients}\label{nutrients}

The importance of nutrients in the functioning of kelps is well
understood {[}@Dayton1985; @Gaylord2012{]}. Dissolved nitrogen, and in
particular nitrate, are important; however research has also placed
emphasis on phosphate and other trace compounds for functioning of kelps
{[}@Dayton1985{]}. Additionally, some kelps have the ability to store
inorganic nitrogen in order to compensate for periods of low nutrient
availability, which has been observed for Laminaria and Macrocystis
{[}@Dayton1985; @Gaylord2012{]}. Nutrient stratification is also an
important factor, particularly for canopy type kelps. The concentration
of nutrients at the surface is important to the functioning and
maintenance of the canopy. For instance kelp canopies in California
often deteriorate in the summer months when surface nitrate levels are
low {[}@Jackson1977{]}. Water motion is important in the assimilation of
nutrients from the water column, and kelps have been shown to adapt
blade morphology in order to create more turbulence around the boundary
layer of the frond to enhance nutrient assimilation (Wheeler 1980).
Temperature has also been closely linked with nutrient concentrations.
Nutrients are often in higher concentrations in the water column during
low temperature events. This is often an indication of an ``up-welling''
event, which brings cold and nutrient rich waters from the bottom to the
surface of the water column. Temperature can play a direct role in the
uptake of nutrients through effects on algal metabolism; however this may
vary from species to species {[}@Raven1988{]}.

\section{Temperature}\label{temperature}

Temperature is a driver of kelp species distributions and
ecophysiological processes, as well as a lesser role in morphological
adaption\ldots{}\textbf{example here}\ldots{}The majority of kelp
species are artic and temperate organisms, and the warming of ocean
temperatures is expected to cause a poleward biogeographical shift of
species {[}@Bolton2012{]}. There is evidence to suggest that South
African kelp forests are expanding due to ocean cooling
{[}@Bolton2012{]}, possibly driven by an intensification and increase in
coastal upwelling {[}@Blamey2012, @Blamey2015{]}. In South Africa there
has been a biogeographical shift eastward along the coast due to a
change in inshore temperature regime, making South Africa no exception
to changing ocean temperatures {[}@Bolton2012{]}. Macroalgae, such as
kelps, can react to an increase in surface temperatures in one of three
ways: they can migrate, adapt and die {[}@Biskup2014{]}. A study by
@Biskup2014 investigated the functional response of two kelp species
(\emph{Laminaria ochroleuca} and \emph{Saccorhiza polyschides}) to
rising sea temperatures. The functional responses of Saccorhiza
polyschides was measured for both the subtidal and intertidal habitats,
to see what affect non- optimal conditions (intertidal zone) had on the
kelps {[}@Rinde2005{]}. The study found that Laminaria ochroleuca
exhibited a poor ability to acclimatise and was dependent on the kelp's
life history traits {[}@Biskup2014{]}. Therefore annual kelp species are
more likely to survive under non-ideal condition, and the intertidal
Saccorhiza polyschides, compared to the subtidal, showed a higher
physiological flexibility to changing conditions {[}@Biskup2014{]}. This
may be because the intertidal zone undergoes far more change than the
subtidal and therefore kelps in the intertidal are forced to adapt to
harsher conditions where fluctuations in temperature, sunlight, turbidity
and water motion are common. The effects on temperature have also been
investigated by @Wernberg2010. The study looked at resilience of kelp
beds along a latitudinal temperature gradient. Kelp abundance is likely
to decline with the predicted warming of ocean waters @Wernberg2010 and
although kelps have the ability to acclimatize and adjust their
metabolic performance, which in turn allows them to change their
physiological performance to mitigate the seasonal fluctuations in
temperature, this acclimatization is done at a cost
@Wernberg2010\ldots{}\textbf{link to paragraph on kelp
morphology}\ldots{}

\section{Wave exposure}\label{wave-exposure}

Other than temperature, wave exposure is also recognised as an important
driver of the marine environment, and macroalgae are not exception. Wave
exposure has been shown to play a role in determing distribution,
abundance, diversity, composition, growth {[}@Cousens1982{]} and
productivity {[}@pedersen2012a{]} of macroalgae communties. For example,
the width, vertical zonation and diversity of algal communities often
change predictably along gradients of wave exposure. Wave exposure may
also drive macroalgae communties indirectly through the alteration in
affect of another environmental driver. For instance, increasing degrees
of exposure may positively influence the amount of area available to
trap light on macroalgal fronds, as well as increasing nutrient uptake
due to increased turbulence in the boundry layer around the frond
{[}@lobban1994{]}. The most important direct effect of wave exposure on
macroalgal communities is through mechanical dislodgment, which
ultimately leads to expiration. Wave exposure is a complex abiotic
variable which varies spatially and temporarly in the marine enviroment.
Furthermore, the degree to which a macroalgae community is exposed which
depend on local site characteristics, such as bathymetry and local wind
patterns. Despite this fact, macrolagae have been able to persist in
often harsh and variable wave environments. Macroalage are sessile
organisms and incapabable of migrating when local conditions become
unsuitable. Therefore, macroalgae must adapt to the local wave climate
in order to persist and survive, and achieve this through morphological
variation.

\begin{quote}
keeping this for now @moss1948 investigated the anatomy, chemical
composistion of \emph{Fucus spiralis} at three sites that varied in wave
exposure (sheltered, medium exposure and exposed). The authors found
that individuals in exposed sites showed less branching of thalli as
well as variation in organic nitrogen, mannitol, laminarian and alginic
acid concentrations.
\end{quote}

\section{Hydrodynamic modelling}\label{hydrodynamic-modelling}

Traditional ecological measures of wave exposure usually incorportates
integrative measures of hyrdodynamic conditions at a particular site.
More specifically, it is the integration of mechanical processes and the
influence that is has on the ecology of nearshore communities.

Wave exposure may be modelled through various methods which range from
simple cartographic to more advanced numerical wave models.
Cartographical models can be qualitative or quantitative and were
designed for the need of wave exposure measures to explain ecological
distributions. A simple set of calculations on coastline and wind data,
and relatively small input data sets are required. These are regarded as
``fetch-based models'', which measure the length of open water
associated with a site along a straight line. The output of such an
approach is a simplified estimate of the potential wave energy for a
specific set of sites. Advances in cartographical methods using
fetch-based models has allowed for wave exposure measurments for larger
areas, and has been suggested as a method for predicting macroalgal
community structure {[}@burrows2008{]}. An example of such a model is
the ``BioEx model'' which was developed by @baardseth1970 to estimate
wave exposure over large regions. BioEx requires frequency, strength and
direction of winds, weighted by degree of exposure within various
directions. BioEx is calculated as the sum of the index developed at
different spatial scales (local, fjord and open). This method has been
used in mapping of marine coastal biodiveristy {[}@Rinde2004{]}.

@lindegarth2005 critized this approach, arguing that the choice of wave
exposure method can influence ecological inference. The authors also
highlighted the need for objective, reproducible and quantitative
studies comparing exposure indices {[}@lindegarth2005{]}. A study by
@sundbald\ldots{}\textbf{include description of study here}\ldots{}Other
authors, such as @hill2010, have argued that these simple measures can
be improved upon by including bathymetery data which allows the
incorporation of diffraction into the calculation. Diffraction is
topographically induced variations in wave direction. A model
incorporating this complexity was developed by @isaeus2004, and is known
as the ``simplified wave model'' (SWM). The model uses measurements of
wind strength, fetch and empirically derived algorithims to mimic
diffraction.

Advances in numerical modelling have been founded on physical wave
theory on how a wave ``behaves''. This approach is based on a
theoretical persepctive rather than the need to answer ecological
questions. Besides diffraction, these models incorporate more complexity
by including wind forcing, wave-to-wave interactions and loss of energy
due to friction and wave breaking. Numerical models have a variety of
applications and are often incorporated within hydrodynamic general
circulation models and are used operationally for forecasting the sea
state {[}@hasselmann1988; @booij1999; @smith2001{]}. The downside of
advanced numerical models is that are computationally intensive which
creates limitations for large scale simulations. Therefore, their
application along long stretches of variable coastline, inshore
environments and ocean-wide simulations limited due to the poor spatial
coverage. However, this models can be designed for local or site
specific coverage, provided the correct data is available.

\subsection{Aims of research}\label{aims-of-research}

The aim of the project is to investigate coastal flow regimes along the
west coast and south-west coast of South Africa and the role this may
play in transport of kelp beach-cast and microplastics. This aim will be
met through the following objectives:

\begin{enumerate}
\def\labelenumi{\Alph{enumi})}
\tightlist
\item
  Determine if the hydrodynamic environment is the main driver of kelp
  morphology and if this is specific to a location
\item
  Simulate kelp rafting by means of a hydrodynamic modelling and
  calibrate this model with \emph{in situ} beach-cast morphometric data.
\item
  Use the calibrated hydrodynamic model to investigate dispersal of
  microplastics along the west coast and south-west coast of South
  Africa.
\end{enumerate}


\end{document}

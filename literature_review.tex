\documentclass[]{article}
\usepackage{lmodern}
\usepackage{amssymb,amsmath}
\usepackage{ifxetex,ifluatex}
\usepackage{fixltx2e} % provides \textsubscript
\ifnum 0\ifxetex 1\fi\ifluatex 1\fi=0 % if pdftex
  \usepackage[T1]{fontenc}
  \usepackage[utf8]{inputenc}
\else % if luatex or xelatex
  \ifxetex
    \usepackage{mathspec}
  \else
    \usepackage{fontspec}
  \fi
  \defaultfontfeatures{Ligatures=TeX,Scale=MatchLowercase}
\fi
% use upquote if available, for straight quotes in verbatim environments
\IfFileExists{upquote.sty}{\usepackage{upquote}}{}
% use microtype if available
\IfFileExists{microtype.sty}{%
\usepackage{microtype}
\UseMicrotypeSet[protrusion]{basicmath} % disable protrusion for tt fonts
}{}
\usepackage[margin=1in]{geometry}
\usepackage{hyperref}
\hypersetup{unicode=true,
            pdftitle={A Transdiciplinary approach to developing a spectral numerical coastal-flow model based on passive macroalgal rafting phenenomea to shed light on ambient and stochastic dispersal and connectivity of nearshore and offshore marine communities in and around the coast of South Africa: Combining ecology and 3D hyperspectral hydrodynamic modelling.},
            pdfauthor={Ross Coppin},
            pdfborder={0 0 0},
            breaklinks=true}
\urlstyle{same}  % don't use monospace font for urls
\usepackage{graphicx,grffile}
\makeatletter
\def\maxwidth{\ifdim\Gin@nat@width>\linewidth\linewidth\else\Gin@nat@width\fi}
\def\maxheight{\ifdim\Gin@nat@height>\textheight\textheight\else\Gin@nat@height\fi}
\makeatother
% Scale images if necessary, so that they will not overflow the page
% margins by default, and it is still possible to overwrite the defaults
% using explicit options in \includegraphics[width, height, ...]{}
\setkeys{Gin}{width=\maxwidth,height=\maxheight,keepaspectratio}
\IfFileExists{parskip.sty}{%
\usepackage{parskip}
}{% else
\setlength{\parindent}{0pt}
\setlength{\parskip}{6pt plus 2pt minus 1pt}
}
\setlength{\emergencystretch}{3em}  % prevent overfull lines
\providecommand{\tightlist}{%
  \setlength{\itemsep}{0pt}\setlength{\parskip}{0pt}}
\setcounter{secnumdepth}{0}
% Redefines (sub)paragraphs to behave more like sections
\ifx\paragraph\undefined\else
\let\oldparagraph\paragraph
\renewcommand{\paragraph}[1]{\oldparagraph{#1}\mbox{}}
\fi
\ifx\subparagraph\undefined\else
\let\oldsubparagraph\subparagraph
\renewcommand{\subparagraph}[1]{\oldsubparagraph{#1}\mbox{}}
\fi

%%% Use protect on footnotes to avoid problems with footnotes in titles
\let\rmarkdownfootnote\footnote%
\def\footnote{\protect\rmarkdownfootnote}

%%% Change title format to be more compact
\usepackage{titling}

% Create subtitle command for use in maketitle
\providecommand{\subtitle}[1]{
  \posttitle{
    \begin{center}\large#1\end{center}
    }
}

\setlength{\droptitle}{-2em}

  \title{A Transdiciplinary approach to developing a spectral numerical
coastal-flow model based on passive macroalgal rafting phenenomea to
shed light on ambient and stochastic dispersal and connectivity of
nearshore and offshore marine communities in and around the coast of
South Africa: Combining ecology and 3D hyperspectral hydrodynamic
modelling.}
    \pretitle{\vspace{\droptitle}\centering\huge}
  \posttitle{\par}
    \author{Ross Coppin}
    \preauthor{\centering\large\emph}
  \postauthor{\par}
    \date{}
    \predate{}\postdate{}
  

\begin{document}
\maketitle

\hypertarget{summary}{%
\section{Summary}\label{summary}}

Marine ecosystems are maintained by a variety of complex interactions
between abiotic and biotic variables such as temperature, wave exposure,
pH, competition, and processes such as top-down and bottom-up control,
predator-prey relationships and phenology (Doney et al.~2012; Harley et
al.~2006; Burrows et al.~2011; McGowan et al.~1998). These abiotic and
biotic variables, the interactions between them, and the various
ecological processes, ultimately determine the community composition and
ecological functioning of all ecosystems (Guimaraes \& Coutinho 1996;
Harley et al.~2006; Poloczanska et al.~2013; Jennings \& Brander 2010;
Polovina 2005; Johnson et al.~2011; Wernberg et al.~2016; Dayton et
al.~1999). Climate directly and indirectly affects the way in which
abiotic and biotic variables interact, but is often compounded by other
impacts such as habitat destruction, pollution, and overfishing (Blamey
et al.~2012; Blamey et al.~2015). Temperature and wave exposure have
been recognised as important variables with regards to climate-driven
changes within the ocean (Guimaraes \& Coutinho 1996; McGowan et
al.~1998; McQuaid \& Branch 1984; Laufkötter et al.~2015; Belkin 2009;
Filbee-Dexter et al.~2018; Miller et al.~2011; Smale et al.~2011; Smale
\& Moore 2017). In order to persist and survive within variable and
changing environments, organisms must either migrate, adapt, or die.

Seaweeds are sessile organisms which are unable to migrate to new areas
when local environmental conditions become unsuitable and therefore are
forced to adapt to new conditions in order to avoid expiration (Dayton
1985; Tegner et al.~1997). The main form of mortality for seaweeds is
through mechanical dislodgement by wave action (Schiel et al.~2006;
Seymour et al.~1989; Thomsen et al.~2004; Cavanaugh et al.~2011; Graham
et al.~1997; Demes et al.~2013; Edwards \& Estes 2006). Seaweeds,
particularly brown seaweeds, are able to undergo rapid morphological
adaptation to the hydrodynamic environment (Miller et al.~2011;
Friedland \& Denny 1995; Wing et al.~2007; Wernberg \& Thomsen 2005;
Fowler-Walker et al.~2006; Dudgeon \& Johnson 1992; Blanchette et
al.~2002). This allows seaweeds to reduce mortality through mechanical
dislodgement by inducing morphology which reduces overall drag. Seaweeds
that are unable to avoid mechanical dislodgement either raft out to sea
or wash up onto beaches. Not all beach-cast kelp may have originated
from a nearby kelp population and may have originated from other sites
or regions of the coast through rafting and ocean currents (Malm et
al.~2004; Emery \& Tschudy 1941; Filbee-Dexter \& Scheibling 2012;
Hobday 2000). Therefore, because kelp morphology is specific to its
local environment the morphological features may be able to indicate,
within a certain amount of probability, what site or region it most
likely originated from. In other words, beach-cast kelp may be used as
proxy for investigating the flow of coastal currents. Using kelp as a
proxy for determining its original location will be calibrated by means
of a hydrodynamic model which will be designed from already existing
SWAN and Delft3D models. This combined approach will allow investigation
into flow regimes around the west and south-west coasts of South Africa
and the role they play in subtidal ecology. Furthermore, the model could
be used to determine transport of plastic pollution around the coast.
For instance, micro-plastics are recognised as a threat to marine life
(Seltenrich 2015; Azzarello \& Van Vleet 1987; Wright et al.~2013;
Derraik 2002; Andrady 2011; Fendall \& Sewell 2009; Ivar Do Sul \& Costa
2014; O 'donoghue \& Marshall n.d.; Ballent et al.~2012), however very
little is known about how microplastics may be transported along the
coast.

The increased use of plastic in society over the past half century has
resulted in large amounts of plastic litter in both the marine and
terrestrial environment (Andrady 2011; Wright et al.~2013; Cole et
al.~2011). The problems associated with large plastic debris have
received attention for many decades, whereas those connected to marine
micro-plastics comparatively received very little attention. However,
today it has become a prioritized area among political organizations,
agencies and NGOs around the world. The micro-plastic debris present in
the ocean are derived from marine and terrestrial sources (Derraik 2002;
Wright et al.~2013; Cole et al.~2011; Seltenrich 2015), however there is
little understanding of how microplastics may be distributed with ocean
currents. Therefore, the coupling the adaptability of kelp morphology
and the ability to simulate hydrodynamic processes can greatly improve
our understanding of transport pathways and likely locations of
accumulation. This in turn may inform management decisions with regards
to eliminating and managing marine pollution in South Africa.

\hypertarget{background}{%
\section{Background}\label{background}}

Seaweeds, browns in particular, are capable of adapting morphological
characteristics to persist in changing and variable ocean environments.
Changes in morphology have been shown to alter photosynthetic ability,
nutrient uptake, and reduce probability of dislodgement with regards to
variation and changes in sea temperature , wave exposure and light
availability. Wave exposure has been shown to be an important driver of
seaweed morphology, as the main mechanism of seaweed mortality is
through the dislodgement. Changing morphology reduces drag and increases
the probability of survival. However, locally adapted seaweed may still
be dislodged in pulse disturbance events such as storms, and may raft
far distances and wash up on beaches. Beach-cast may not always
originate from adjacent kelp populations but rather from other regions
which the individual is adapted for. Therefore, the kelp morphology may
act as a proxy for investigating coastal currents and changes thereof.
The advances in ocean hydrodynamic modelling has made great progress and
has been applied in a variety of ways. This study will use advances in
hydrodynamic modelling in combination with kelp morphological
characteristics to investigate coastal currents. Once the model has been
established it may be applied in other ways, such as investigating the
transport of microplastics along the South African coastline. The harms
of microplastics to the marine environment has gained much traction in
recent years, but research in South Africa is lacking. The coupling of
kelp morphology and the ability to simulate hydrodynamic processes can
greatly improve our understanding of transport pathways and likely
locations of accumulation. This in turn may inform management decisions
with regards to eliminating and managing marine pollution in South
Africa.

Pollution is a huge environmental problem that affects both terrestrial
and marine ecosystems. Pollution from land enters the sea where it could
harm or kill marine organisms or is transported by ocean currents to
other coastal areas. In these areas the pollution could re-enter the
ocean or be blown by wind into terrestrial ecosystem where it could once
again be ingested by organisms causing harm or even death. Pollution is
therefore a major threat to the environment and marine organisms. In
recent years the effect of microplastics on the ocean has gained much
traction with scientists and politicians alike. Microplastics cosiste of
tiny particles of plastic and other pollution and are therefore are
difficult to detect. Currently there is no hydrodynamic model that is
able to determine dispersion, source and accumulation of microplastics
along the South African coastline. Given the significant detrimental
effect microplastics play in the ocean, it is important that such a
mechanism be developed that will aid in better management of marine
pollution in South Africa. Furthermore, this project allows for a
multidisciplinary approach to be taken by combining ecology and coastal
oceanography.

Abiotic and biotic factors interact in complex ways which indirectly
determine behavioral and ecophysiological responses in organisms. For
example, when storms or strong currents form in sub-tidal habitats, sea
urchins form aggregations in order to reduce overall drag to avoid being
swept away by currents. Organisms that are motile my migrate into more
environmentally suitable areas when conditions become unfavorable or
food sources become depleted. In changing environments migration may
also allow organisms to extend their distributions. For example, ocean
warming off the coast of Western Australia has allowed tropical fish
species to extend their distribution into areas that were previously
characterised as temperate reefs. Sessile organisms are unable to
migrate into more environmentally suitable areas and are forced to
either adapt or suffer expiration. Sessile organisms may respond to
changing environmental conditions through changes in physiology. For
example, plants may produce heat shock proteins that help buffer the
effect of temperature increases. Sessile organisms may also adapt their
morphology,in order to persist in changing and harsh environments, such
as seaweeds.

Temperature and wave exposure have been shown to be important drivers of
seaweed distribution, physiological functioning, ability to recover,
population dynamics and morphology. Mechanical forces generated by the
hydrodynamic environment, in the form of sudden strong ocean currents or
storms, between 10- 20 m s-1 with accelerations of 400 m s-2 (Friedland
and Denny 1995) are the biggest threat to kelp survival. Kelps are able
to rapidly adapt their morphological characteristics to reduce drag and
avoid dislodgement (Blanchette 1997). For example a study by Koehl et
al.~(2008) showed that transplanted Nereocystis luetkeana plants from a
wave sheltered site to a wave exposed site changed their morphology to
flat blades and narrow laterals that are less prone to drag forces in
4-5 days. Another study by Fowler-Walker, Wernberg, and Connell (2006)
tested for differences in morphology of Ecklonia radiata between
wave-sheltered and wave-exposed sites and through a combination of in
situ sampling and transplantation of juvenile plants. The results showed
that morphology differed between wave-sheltered and wave-exposed sites
(thin thallus at sheltered sites and a narrow, thick thallus with a
thick stipe at exposed sites), and was consistent with previous studies.
Juveniles transplanted into wave exposed sites under went rapid
morphological adaptation, whilst the opposite was true for
wave-sheltered sites which showed slower morphological adaptation.

Kelp morphology may be distinct to a particular region with a specific
hydrodynamic environment and has the ability to raft far distances using
coastal currents, and may accumulate as beach-cast in areas far from its
original location. Therefore, kelp morphology may be used as a proxy for
determining its original location as well as aid in characterising
coastal currents. However, this approach must be combined with advances
in hydrodynamic modelling for a quantitative outcome.

Advances in numerical modelling has gained much traction in recent years
and has been applied in a variety of ways with regards to ecological
studies. For example, a study by Wang and Xia (2009) used the
Delft3D-Flow model to assess the hydraulic suitability of a stream as a
spawning ground for the Chinese Sturgeon (Acipenser sinensis) in the
Yangtze River. The authors calculated the horizontal mean vorticity
which was used to assess the hydraulic environment of spawning ground.
The flow field state was determined through model simulation and
field-measured data used to validate the model. The results added to
existing scientific database for spawning ground hydraulic environmental
protection. Different numerical models can often be integrated to model
across ecosystem levels. For example a study by Leon et al.~(2003) used
integrated physical (Delft3D hydrological model) and bio-chemical
(Agricultural Non-point Source model) processes models to investigate
the possible impact on the Lake Malawi water quality due to management
actions performed at the watershed level.

Since wave energy is an important driver in marine ecosystems,
particularly kelp, the advances in hydrodynamic modelling offer a new
opportunity for multifactorial and quantitative approach to research in
marine ecosystems. The Delft3D and SWAN models have been used
successfully in previous studies regarding brine plume discharge,
impacts of storms, effects of climate change on the hydrological
environment etc. The models have not been designed for shallow
environments (\textless{}6m) and therefore may not be suitable to model
coastal hydrological environments. However these models may be adjusted
to suit coastal waters if they are combined with a new numerical model
which can be calibrated to suit these needs.

In recent years there has been growing attention on plastic pollution,
particularly in the ocean. Plastic pollution can be in the form of
macro- and microplastics. Microplastics are tiny plastic granules used
as scrubbers in cosmetics and air-blasting, and small plastic fragments
that originate from larger pieces of plastic known as macroplastics,
while macroplastics\ldots{}insert definition here\ldots{} The potential
harms of of plastic pollution in the marine environment was highlighted
in the 1970's and renewed interest has lead to research showing that
plastic pollution in the ocean are widespread. Plastics may become
bio-available to biota in the food-web which may cause problems with an
organism's physiological functioning. Furthermore, the relatively large
surface area and composition of microplastics provides an environment
that is able of adhering to organic pollutants. In other words
microplastics also act as a vector for transport and assimilation of
organic pollutants.

Therefore, this study not only enables research into the ecological
effects of the hydrological environment on an important habitat-forming
organism, it also offers the opportunity to improve on current
hydrological numerical models to suit coastal environments. This in turn
will allow investigation into the flow and accumulation of microplastics
which are regarded as a major threat to marine life. Furthermore, the
calibrated model could be applicable to other ecological studies such as
dispersal of benthic flora and fauna, climate change studies,
forecasting as some examples.

\hypertarget{kelp-environmental-drivers}{%
\section{Kelp environmental drivers}\label{kelp-environmental-drivers}}

Important environmental drivers of kelp individuals and communities
include light, substrata, salinity, sedimentation, nutrients,
temperature and wave exposure. Although studies have investigated the
effects of these important environmental drivers, the roles these
factors play is often difficult to evaluate as such factors may never be
fully independent of each other, i.e.~environmental factors are to some
extent are dependent on one another. Multifactorial studies have
attempted to explain combined effects, however these studies are often
limited to investigating combination of two or three environmental
drivers as inclusion of too many factors can lead to results that are
difficult to interpret. Environmental factors are highly variable on
temporal and spatial scales and their effects may also be dependent on
the life-stage of the organism, adding a further layer of complexity to
investigations.

\hypertarget{light}{%
\subsection{Light}\label{light}}

Light is an important factor for kelp survival, however if light is
limited or excessive this may negatively impact kelp survival or growth.
Much of the past research into the role light plays into the functioning
of kelp (Bruhn and Gerard 1996; ???). For instance, solar ultraviolet
radiation has been shown to affect sub-canopy Ecklonia radiata
sporophytes when the canopy of mature Ecklonia radiata was removed (Wood
1987). The sub-canopy sporophytes experienced tissue damage,photopigment
destruction,reduced growth and decreased survivorship, thus inhibiting
their settlement and survival (Wood 1987). Laboratory experiments
revealed that the UV component of radiation, rather than intense
radiation itself, was responsible for the effects mentioned above. High
light stress has negative effects, such as photoinhibition and
photo-damage on Ecklonia cava sporophytes (Altamirano and Murakami
2004). Altamirano and Murakami (2004) found that Ecklonia cava is more
vulnerable to light stress conditions, and less likely to recover under
unfavourable conditions (Altamirano and Murakami 2004). Bolton and
Levitt (1985) showed that under sub-saturating irradiances and supra-
optimal temperatures Ecklonia maxima to showed a decrease in
reproductive rates and an increase in cell production. An additional
finding of this study was that despite the decrease in reproductive
rates, the final egg production per female was greater under these
conditions. The authors interpreted this an ecological adaptation that
may increase survival rates under times of stress or non - ideal
conditions (Bolton and Levitt 1985).

\hypertarget{substrata}{%
\subsection{Substrata}\label{substrata}}

\hypertarget{salinity}{%
\subsection{Salinity}\label{salinity}}

\hypertarget{depth}{%
\subsection{Depth}\label{depth}}

Depth does not affect kelp ecosystems directly, however a change in
depth causes fluctuations or changes in other environmental variables
such as water motion, light and temperature. Water motion also decreases
with depth, and some kelps better suited to deeper environments
(\emph{L. pallida}) replace those in the shallows (\emph{E. maxima})
(Dayton 1985; Gerard 1982). The increase in depth can lead to a decrease
in sunlight penetration, with some species better adapted for low-light
conditions than others, such as (\emph{L. pallida}). Temperature may
also change along a depth gradient due to a reduction in sunlight
penetration (Dayton 1985; Gerard 1982). Therefore depth does not
directly play a role in kelp functioning but may alter more influential
factors such as light and water motion.

\hypertarget{sedimentation}{%
\subsection{Sedimentation}\label{sedimentation}}

\hypertarget{nutrients}{%
\subsection{Nutrients}\label{nutrients}}

The importance of nutrients in the functioning of kelps is well
understood (Dayton 1985; Gaylord, Nickols, and Jurgens 2012). Dissolved
nitrogen, and in particular nitrate, are important; however research has
also placed emphasis on phosphate and other trace compounds for
functioning of kelps (Dayton 1985). Additionally, some kelps have the
ability to store inorganic nitrogen in order to compensate for periods
of low nutrient availability, which has been observed for Laminaria and
Macrocystis (Dayton 1985; Gaylord, Nickols, and Jurgens 2012). Nutrient
stratification is also an important factor, particularly for canopy type
kelps. The concentration of nutrients at the surface is important to the
functioning and maintenance of the canopy. For instance kelp canopies in
California often deteriorate in the summer months when surface nitrate
levels are low (Jackson 1977). Water motion is important in the
assimilation of nutrients from the water column, and kelps have been
shown to adapt blade morphology in order to create more turbulence
around the boundary layer of the frond to enhance nutrient assimilation
(Wheeler 1980). Temperature has also been closely linked with nutrient
concentrations. Nutrients are often in higher concentrations in the
water column during low temperature events. This is often an indication
of an up-welling event, which brings cold and nutrient rich waters from
the bottom to the surface of the water column. Temperature can play a
direct role in the uptake of nutrients through effects on algal
metabolism; however this may vary from species to species (???).

\hypertarget{temperature}{%
\subsection{Temperature}\label{temperature}}

Temperature is a driver of kelp species distributions and
ecophysiological processes, as well as a lesser role in morphological
adaptation\ldots{}example here\ldots{}The majority of kelp species are
arctic and temperate organisms, and the warming of ocean temperatures is
expected to cause a poleward biogeographical shift of species (Bolton et
al.~2012). There is evidence to suggest that South African kelp forests
are expanding due to ocean cooling (Bolton et al.~2012), possibly driven
by an intensification and increase in coastal upwelling (Blamey and
Branch 2012, Blamey et al.~(2015)). In South Africa there has been a
biogeographical shift eastward along the coast due to a change in
inshore temperature regime, making South Africa no exception to changing
ocean temperatures (Bolton et al.~2012). Macroalgae, such as kelps, can
react to an increase in surface temperatures in one of three ways: they
can migrate, adapt and die (Biskup et al.~2014). A study by Biskup et
al.~(2014) investigated the functional response of two kelp species
(Laminaria ochroleuca and Saccorhiza polyschides) to rising sea
temperatures. The functional responses of Saccorhiza polyschides was
measured for both the subtidal and intertidal habitats, to see what
affect non- optimal conditions (intertidal zone) had on the kelps (Rinde
and Sjøtun 2005). The study found that Laminaria ochroleuca exhibited a
poor ability to acclimatise and was dependent on the kelp’s life
history traits (Biskup et al.~2014). Therefore annual kelp species are
more likely to survive under non-ideal condition, and the intertidal
Saccorhiza polyschides, compared to the subtidal, showed a higher
physiological flexibility to changing conditions (Biskup et al.~2014).
This may be because the intertidal zone undergoes far more change than
the subtidal and therefore kelps in the intertidal are forced to adapt
to harsher conditions where fluctuations in temperature, sunlight,
turbidity and water motion are common. The effects on temperature have
also been investigated by Wernberg et al.~(2010). The study looked at
resilience of kelp beds along a latitudinal temperature gradient. Kelp
abundance is likely to decline with the predicted warming of ocean
waters Wernberg et al.~(2010) and although kelps have the ability to
acclimatize and adjust their metabolic performance, which in turn allows
them to change their physiological performance to mitigate the seasonal
fluctuations in temperature, this acclimatization is done at a cost
Wernberg et al.~(2010)\ldots{}link to paragraph on kelp
morphology\ldots{}

\hypertarget{wave-exposure}{%
\subsection{Wave exposure}\label{wave-exposure}}

Other than temperature, wave exposure is also recognised as an important
driver of the marine environment, and macroalgae are not exception. Wave
exposure has been shown to play a role in determining distribution,
abundance, diversity, composition, growth (Cousens 1982) and
productivity (Pedersen and Nejrup 2012) of macroalgae communities. For
example, the width, vertical zonation and diversity of algal communities
often change predictably along gradients of wave exposure. Wave exposure
may also drive macroalgae communities indirectly through the alteration
in effect of another environmental driver. For instance, increasing
degrees of exposure may positively influence the amount of area
available to trap light on macroalgal fronds, as well as increasing
nutrient uptake due to increased turbulence in the boundary layer around
the frond (???). The most important direct effect of wave exposure on
macroalgal communities is through mechanical dislodgement, which
ultimately leads to expiration. Wave exposure is a complex abiotic
variable which varies spatially and temporarily in the marine
environment. Furthermore, the degree to which a macroalgae community is
exposed, is dependent on local site characteristics, such as bathymetry
and local wind patterns. Despite this fact, macroalgae have been able to
persist in often harsh and variable wave environments. Macroalgae are
sessile organisms and incapabable of migrating when local conditions
become unsuitable. Therefore, macroalgae must adapt to the local wave
climate in order to persist and survive, and achieve this through
morphological adaptation. The morphology of macroalgae are not fixed
genetic traits. A study by Koehl et al.~(2008) showed that transplanted
Nereocystis luetkeana plants from a wave sheltered site to a wave
exposed site changed their morphology to flat blades and narrow laterals
that are less prone to drag forces in 4-5 days. Advances in genetic
techniques and taxonomy have revealed that species delineation based on
morphology has been inaccurate, and organisms that were once considered
two separate species are actually one species. For example, Moss (1948)
investigated the anatomy, chemical composition of Fucus spiralis at
three sites that varied in wave exposure (sheltered, medium exposure and
exposed). The authors found that individuals in exposed sites showed
less branching of thalli as well as variation physiological components,
such as organic nitrogen, mannitol, laminarian and alginic acid
concentrations. The authors also noted a `crumpling effect' displayed by
individuals from exposed sites and inferred that this strategy may
reduce overall drag. Other studies show that macroalgae in wave exposed
environments have morphologies that reduce overall drag, increase
strength of attachment or increase flexibility. There is also evidence
that morphological adaptation is driven by currents, and in fact may be
driving hydrological performance of macroalgae. Duggins et al.~(2003)
examined the direct and indirect flow effects on population dynamics,
morphology and biomechanics of several understorey macroalgae species.
These species included Costaria costata, Agarum fimbriatum, and
Laminaria complanata and * Nereocystis luetkeana*. The results showed
that in wave impacted sites (wave exposed) had higher rates of
mortality, and no significance was found between survival of individuals
and tidal or current velocity. The authors concluded that although tidal
and current velocity did not play a significant role in determining kelp
survival, it did play a role in morphological adaptation. The results
from this study suggest that high current and tidal stresses are the
main driver of kelp morphological adaptation. This in turn make those
individuals more resilient to dislodgement to wave exposure. Wave
exposure is stochastic in nature compared to tidal and ocean currents
which are more regular in their frequency and magnitude. Therefore the
regular forces of tidal and ocean currents may make kelp individuals
more resilient to mechanical dislodgement over time.

The morphological adaptation that macroalgae display are therefore
driven by site conditions, therefore individuals must be morphologically
flexible to persist in stochastic environments. This may be achieved
through different strategies and are species-specific, which can be
directly attributed the high diversity in morphological characters of
algal communities. For example some algae have fronds and others are
articulated coralline, and therefore these species would need to adapt
their morphology differently in order to persist. In general, flat
strap-like blades are common in areas that are exposed to high wave
energy, while at protected sites blade morphology is wide and undulated.

\hypertarget{ocean-and-coastal-waves}{%
\section{Ocean and coastal waves}\label{ocean-and-coastal-waves}}

\hypertarget{introduction}{%
\subsection{Introduction}\label{introduction}}

Regardless of the location around the world, waves are a feature of any
coastline and the marine environment and are important manifestations of
energy in the ocean. Waves are not the movement of water particles but
is rather the movement and propagation of energy through the ocean. The
source of this energy can be formed locally, such as wind-driven waves,
and or from distant locations in the ocean, such as storms. This energy
is then transferred from deeper water into shallower water where it
plays a role in driving beach morphology. Sand and rock are eroded with
each passing wave, thereby playing a role in shaping coastlines. Energy
that is left-over after these processes is transferred into heat energy,
and heats up the sand and rocks on which each wave hits. This energy can
also be harnessed in simple ways such as a surfer catching a wave, or
more complex ways such as capturing energy from the ocean environment
for electricity production. However, waves are stochastic in nature and
therefore the energy that propagates through the ocean is not always
consistent. Therefore, it is often difficult to quantify and predict
wave energy in the marine environment. Despite this fact, waves are
gaining more recognition for the role they play in shaping coastlines,
beaches and hence the communities and organisms that depend on these
systems.

Waves manifest themselves in different ways, which is also dependent on
the energy creating force, and can be classified into different
categories. For instance, ``chop'' are produced by local winds, while
``tsunamis''" are rare waves that are formed during a earthquakes or
landslides and can be produced from thousands of kilometers away. Waves
in the ocean are known as ``swell'' and are produced from far distant
storms, while the most consistent form of waves that interacts with the
coastlines, usually twice a day, are called ``tides''. Tides are
produced from differences in gravitational forces between the ocean,
earth's crust, sun and the moon. These classifications can be broken
down further and will be investigated later in this chapter. The focus
of this chapter will be understanding wave theory and how waves form and
propagate into shallow-coastal waters. This will be essential to later
chapters that aim to model wave energy and calibrate the produced model
with biotic variables.

\hypertarget{generating-and-restoring-forces}{%
\subsection{Generating and restoring
forces}\label{generating-and-restoring-forces}}

Waves are formed due to the constant interaction between two forces,
which are known as the ``generating source'' and the ``restoration
force''. The generating force is the force which pushes water from one
layer up into the other, for example pushing water up across the
boundary of air. This can also occur in the ocean layers. The ocean
layers through differences in density, which in turn is partly driven by
abiotic parameters such as temperature and salinity, which creates a
boundary for wave energy to move along. Examples of such boundaries are
along the ocean surface, pycnoclines, and the sea-floor. optional
addition of pycnocline details. In other words, waves occur along the
boundaries formed in the ocean by various abiotic processes, and
therefore density boundaries are essential to the propagation of energy
through the ocean.

The generating force and restoring force occur along these boundaries,
the generating force pushes water up across ocean boundaries while the
restoring force pulls it back to where the boundary was originally,
trying to restore the balance in energy. This ``tug of war'' between the
two forces creates an oscillating motion between the boundary layers
which acts as a point of disturbance, sending out energy in all
directions. The disturbance energy will continue to manifest itself
provided the tug and pull between the two forces is occuring. Once the
generating force stops, and ultimately the energy from the point of
disturbance, the waves dissipate and water restores to its original
state. A simple example would be blowing on water in a cup. The blowing
of air onto the surface of the water in the cup creates a point of
disturbance that pushes the air boundary into the water boundary. The
restoring force is the surface tension of the water that is maintained
through hydrogen bonds between water molecules. Once blowing on the
water in the cup has stopped, the ripples or ``waves'' in the cup begin
to diminish in size until the surface tension returns to its original
state. The hydrogen bonds between the water molecules are stronger than
the force of gravity, and therefore the force of the surface tension
returns the water to its original state. These waves are known as
``capillary waves'' and is essentially residual energy after the
generating source has stopped. The restoring force can also be in the
form of gravity and are known as ``gravity waves''. Using the same
example, if one blows too hard, the water boundary is pushed up into the
air boundary, breaking the hydrogen bonds between water molecules which
allows gravity to return the water to its original state. All waves in
the ocean are either capillary waves or gravity waves, and their
classification will be dependent on the restoring force involved.

In nature, there are three kinds of generating forces. These are wind,
displacement of large volumes of water and uneven forces of
gravitational attraction between the Earth, Moon and Sun. Different
generating forces are associated with different wave heights, periods
and the type of wave produced insert appropriate figure. Wind-generated
waves comprise of capillary waves, chop, swell and seiche. Some seiche
can form from landslides and earthquakes but most of these waves are
known as a tsunami. Swell create the waves with the large heights (up to
100m), while tides can create the tallest.

\hypertarget{wave-physics-and-scales}{%
\subsection{Wave physics and scales}\label{wave-physics-and-scales}}

Waves have a number of characteristics which are depicted in figure ??,
and is an idealised representation of what a wave is. The amplitude is
the vertical distance from its midline or equilibrium surface or still
water level to its highest point known as the crest. The equilibrium
surface is the level the ocean would be if there were no waves, for a
wave to form a disturbance must occur below or above this line. The
trough is the same distance of the amplitude, however the measurement is
taken from the equilibrium surface to the lowest point. Wave height is
the vertical distance from crest to trough, and is equal to twice the
amplitude. The wavelength is the horizontal distance from a crest/trough
to the next crest/trough respectively. It is important to note that the
energy propagating from a disturbance will not reach the ocean floor in
deep-water environments. The depth below a wave where the water, and
anything in the water, feels no motion or disturbance is known as the
wave base. The wave base is calculated by descending vertically from the
equilibrium surface by a value equal to halve the wavelength. The water
particles in the ocean move in a circular orbit and hence return to
their original position. This is because waves in the ocean represent
moving energy, and not moving water.

\hypertarget{types-of-waves}{%
\subsection{Types of waves}\label{types-of-waves}}

There are a variety of waves that form in the ocean and all differ in
terms of period or wavelength (??image??). The longest wave that can
form in the ocean are known as trans-tidal waves, and are generated by
fluctuations in magnetism between the Earth's crust and atmosphere. The
magnetic push and pull between the Moon and the Sun creates waves with a
slightly shorter wavelength, known as tides. Their period and wavelength
can also range from a few hours to more than a day, and from a few
hundred to a thousand kilometers respectively. Storm surge tend to have
a slightly shorter wavelength and period compared to tides. When low
atmospheric pressure systems and high wind speeds in a storm it elevates
the ocean surface, generating storm surge, which may cause flooding in
coastal areas as it approaches the coastline. Tsunami's are on the lower
end of the scale, and are generated by earthquakes or submarine
`landslides. Their random nature makes them difficult to predict and
increase amplitude as they approach the coast. This can make them
considerably large along coastal areas and often cause immense damage
and loss of life.

\hypertarget{measuring-waves}{%
\subsection{Measuring waves}\label{measuring-waves}}

Waves are often thought of as an elevation of the sea surface from a
specific point over a period of time but this is obvisiously not he
case. This is known as \emph{surface elevation} and is the instaneous
elevation of the sea surface abouve a specific point in a time record
(see figure ???). Although surface elevation does not represent a wave,
it can be used to create a wave profile. This is achieved by profiling
the the surface elevation between two successive \emph{downward
zero-crossings} or \emph{upwardward zero-crossings} (see figure ??).
Both zero-crossings are symmetrical and essentially the same
statistically. However, in practice the downward zero-crossings are
preferred as it takes steepness of a wave into account (the front, see
figure ??) which is relevant to characterising breaking waves. It should
be noted that surface elevation can be negative while a wave profile
cannot.

\begin{figure}
\centering
\includegraphics{literature_review_files/figure-latex/surface elevation figures-1.pdf}
\caption{The definition of a wave in a time record of the surface
elevation with downward zero-crossings (upper panel) and upward
zero-crossings (lower panel).}
\end{figure}

Characterising waves in a wave record requires compromise both in the
statistical sense and the practical, as it is a balance between keeping
the record short enough to remain stationary and long enough for a
reasonable averages to be calculated. The waves are characterised in
terms of wave heights and wave periods for individual waves in the
record and then averaged over that specific time. For example, a time
record of 15-20 minutes is standard when calculating a wave profile.

\begin{figure}
\centering
\includegraphics{literature_review_files/figure-latex/Hs_Tp_surface_elevation figure-1.pdf}
\caption{The definition of wave height and wave period in a time record
of the surface elevation (the wave is defined with downwrd
zero-crossings.}
\end{figure}

Waves are complex and various approaches, techniques and devices have
been developed in order to measure waves effectively. One of the ways
that has been used extensivley in the past but less so today, are visual
estimates which can be used to characterise \emph{significant swell
height} (Hs) and \emph{significant swell period} (Tp). Visual estimates
use 15 - 20 of the most well defined, higher waves of a number of wave
groups to characterise Hs and Tp. Although these parameters are useful,
they do not accurately reflect the waves that are occuring in nature.
Ocean waves are a combination of wind sea (short, irregular, locally
generated waves) and swell (long, smooth waves, generated by distant
storms) and so more parameters are needed to seperate Hs and Tp driven
by wind sea and swell, i.e.~seperate Hs and Tp parameters for wind sea
and swell. Even with seperate parameters for the different types of
waves would still not be enough to effectivley characterise waves in the
ocean. In order to characterise the detail and complexity of ocean waves
a different approach must be used, know as the \emph{spectral} approach.
This approach is based on the idea that the sea surface can be
characterised as the summation of a large number of harmonic wave
components.

\hypertarget{hydrodynamic-modelling}{%
\section{Hydrodynamic modelling}\label{hydrodynamic-modelling}}

\hypertarget{introduction-1}{%
\subsection{Introduction}\label{introduction-1}}

Wave exposure may be modelled through various methods which range from
simple cartographic to more advanced numerical wave models. Traditional
ecological measures of wave exposure usually incorporates integrative
measures of hydrodynamic conditions at a particular site. Cartographical
models can be qualitative or quantitative and were designed for the need
of wave exposure measures to explain ecological distributions. A simple
set of calculations on coastline and wind data, and relatively small
input data sets are required. These are regarded as ``fetch-based
models'', which measure the length of open water associated with a site
along a straight line. The output of such an approach is a simplified
estimate of the potential wave energy for a specific set of sites.
Advances in cartographical methods using fetch-based models has allowed
for wave exposure measurements for larger areas, and has been suggested
as a method for predicting macroalgal community structure (???). An
example of such a model is the ``BioEx model'' which was developed by
Baardseth and others (1970) to estimate wave exposure over large
regions. BioEx requires frequency, strength and direction of winds,
weighted by degree of exposure within various directions. BioEx is
calculated as the sum of the index developed at different spatial scales
(local, fjord and open). This method has been used in mapping of marine
coastal biodiversity (???). Lindegarth and Gamfeldt (2005) critized this
approach, arguing that the choice of wave exposure method can influence
ecological inference. The authors also highlighted the need for
objective, reproducible and quantitative studies comparing exposure
indices (Lindegarth and Gamfeldt 2005). Other authors, such as Hill et
al.~(2010), have argued that these simple measures can be improved upon
by including bathymetry data which allows the incorporation of
diffraction into the calculation. Diffraction is topographically induced
variations in wave direction. A model incorporating this complexity was
developed by Isæus (2004), and is known as the ``simplified wave model''
(SWM). The model uses measurements of wind strength, fetch and
empirically derived algorithms to mimic diffraction.

Advances in numerical modelling have been founded on physical wave
theory on how a wave ``behaves''. This approach is based on a
theoretical perspective rather than the need to answer ecological
questions. Besides diffraction, numerical models incorporate more
complexity by including wind forcing, wave-to-wave interactions and loss
of energy due to friction and wave breaking. Numerical models have a
variety of applications and are often incorporated within hydrodynamic
general circulation models and are used operationally for forecasting
the sea state (Group 1988; Booij, Ris, and Holthuijsen 1999; Smith,
Sherlock, and Resio 2001). The downside of advanced numerical models is
that are computationally intensive which creates limitations for large
scale simulations. Therefore, their application along long stretches of
variable coastline, inshore environments and ocean-wide simulations are
limited due to the poor spatial coverage. However, numerical models can
be designed for local or site specific coverage, provided the correct
data is available. \textbf{\ldots{}More to be added\ldots{}}

\hypertarget{delft-3d-numerical-suite}{%
\subsection{Delft-3D numerical suite}\label{delft-3d-numerical-suite}}

The Delft-3D numerical suite provides an advanced approach to
hydrodynamic modelling through consideration of various physical
phenomena and is a quantitative estimate of wave exposure. The suite of
models can be used for a range of applications such as simulating flow,
sediment transport, waves, water quality, coastal morphological
development and ecology. The suite of numerical models consists of two
modules; Delft-3D WAVE and Delft- 3D FLOW.

\hypertarget{delft-3d-wave}{%
\subsubsection{Delft-3D WAVE}\label{delft-3d-wave}}

The Delft-3D WAVE module uses the SWAN (Simulating Waves and Nearshore)
numerical models to simulate the generation and propagation of
wind-generated waves in coastal environments. The SWAN model is based on
discrete spectral action balanced equation and is fully spectral.
Spectral refers to the consideration of all wave directions and
frequencies and implies that short-crested random wave fields
propagating from different directions can be accounted for. The final
output of the model is wind-sea with superimposed swell.

\hypertarget{delft-3d-flow}{%
\subsubsection{Delft-3D FLOW}\label{delft-3d-flow}}

This module is a multi-dimensional hydrodynamic/transport simulation
program which calculates non-steady flow and transport processes that
result from tidal and meteorological forcing. The dimensions can be
either 2D or 3D and can be placed on a rectilinear or curvilinear,
boundary fitted grid.

\hypertarget{coastal-flow}{%
\section{Coastal flow}\label{coastal-flow}}

\hypertarget{kelp-rafting}{%
\section{Kelp-rafting}\label{kelp-rafting}}

\hypertarget{kelp-drag-dynamics}{%
\section{Kelp-drag dynamics}\label{kelp-drag-dynamics}}

\hypertarget{kelps-in-south-africa}{%
\section{Kelps in South Africa}\label{kelps-in-south-africa}}

The biogeographic distribution of kelp is limited by seawater
temperature (Bolton 2010), where increasing temperature gradients reduce
kelp distribution. Due to this limiting factor, the two main species of
kelps in southern African waters, \emph{Ecklonia maxima} and
\emph{Laminaria pallida}, are distributed along a section of the south
coast from De Hoop, extending west around the Cape Peninsula, and
thriving north into Namibia (Molloy and Bolton 1996, Stegenga et
al.~1997). This distribution follows a temperature gradient, where sea
temperatures increase as one moves south from Namibia, around Cape Point
and towards De Hoop. Although the two species occur together for the
majority of the coast, their basic morphologies and resource needs vary
to a degree. The larger species, \emph{E. maxima}, is distributed from
Lüderitz to Cape Agulhas (Fig. 1) (Bolton and Levitt 1985,Probyn and
McQuaid 1985, Bolton and Anderson 1987, Bolton et al.~2012). The
biogeographic distribution of kelp is limited by seawater temperature
(Bolton 2010), where increasing temperature gradients reduce kelp
distribution. Due to this limiting factor, the two main species of kelps
in southern African waters, Ecklonia maxima and* L. pallida*, are
distributed along a section of the south coast from De Hoop, extending
west around the Cape Peninsula, and thriving north into Namibia (Molloy
and Bolton 1996, Stegenga et al.~1997).

This distribution follows a temperature gradient, where sea temperatures
increase as one moves south from Namibia, around Cape Point and towards
De Hoop. Although the two species occur together for the majority of the
coast, their basic morphologies and resource needs vary to a degree. The
larger species, \emph{E. maxima}, is distributed from Lüderitz to Cape
Agulhas (Fig. 1) (Bolton and Levitt 1985, Probyn and McQuaid 1985,
Bolton and Anderson 1987, Bolton et al.~2012). Characterised by a large
distal swollen bulb filled with gas, and smooth fronds, this species
grows to approximately 10 meters (Bolton and Anderson 1987). There was,
however, a 17-meter specimen collected in 2015 off Cape Point (Smit,
unpubl. data).This species of kelp not only dominate the biomass of the
South African nearshore, but plays an important ecological role
(Bustamante and Branch 1996). The estimated productivity of \emph{E.
maxima} within South Africa varies between 350 and 1500g Cm-2yr-1 (Mann
1982). Across the majority of the coastline, Laminaria pallida remains a
subsurface kelp, dominating the kelp biomass at depths greater than 10
meters (Field et al.~1980a, Bolton and Anderson 1987, Molloy and Bolton
1996). This species is distributed from Danger Point, east of the Cape
Peninsula, to Rocky Point in northern Namibia, and reaches depths of
greater than 20 meters (Field et al.~1980a, Molloy and Bolton 1996,
Stegenga et al.~1997). Towards the north along the west coast, from
around Hondeklipbaai, \emph{L. pallida} replaces \emph{E. maxima} as the
dominant kelp species (Velimirov et al.~1977,Stegenga et al.~1997) and
it also occupies increasingly shallow subtidal regions. The northern
populations also exhibit an increase in stipe hollowness, compared to
the solid stipe morphs in the species' southern distributions (Molloy
and Bolton 1996). This variation in morphology was thought to represent
two distinct species, with the northern populations formerly described
as Laminaria schinzii Foslie (Molloy and Bolton 1996). Genetic work has
subsequently shown that the two morphs are in fact the same species
(Rothman et al.~2017). In southern African waters, the primary
production of \emph{Laminaria pallida} is between 120 and 1900g C m2yr1,
similar to that of \emph{E. maxima} (Mann 1982). Primary production is
not the only pathway.

\hypertarget{kelp-drag-properties}{%
\section{Kelp drag properties}\label{kelp-drag-properties}}

\hypertarget{aims-of-research}{%
\section{Aims of research}\label{aims-of-research}}

The aim of the project is to investigate coastal flow regimes along the
west coast and south-west coast of South Africa and the role this may
play in transport of kelp beach-cast and microplastics. This aim will be
met through the following objectives:

\begin{enumerate}
\def\labelenumi{\arabic{enumi}.}
\item
  Determine if the hydrodynamic environment is the main driver of kelp
  morphology and if this is specific to a location
\item
  Simulate kelp rafting by means of a hydrodynamic modelling and
  calibrate this model with in situ beach-cast morphometric data.
\item
  Use the calibrated hydrodynamic model to investigate dispersal of
  microplastics along the west coast and south-west coast of South
  Africa.
\end{enumerate}


\end{document}
